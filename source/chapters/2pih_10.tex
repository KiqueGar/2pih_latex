\chapter{I Love The Way You Lie (pt. 2)}
‘Twas down the glen, one Easter morn
To a city fair rode I
There armored lines of goblin-kind
In squadrons passed me by
No pipes did hum, no battle drum
Did sound its rare tattoo
While the Angelus bells o’er the Liffey’s swells
Rang out in the foggy dew.

‘Twas wizards bound our ancient lore
So that our nations would not be free.
Their lonely graves at Bas Cliabhan’s waves
On the fringe of the great North Sea
Those who died by Ulak’s side
They stood both tall and true
Their names we shall keep where our fathers sleep,
‘Neath the shroud of the foggy dew.

And the bravest fell while the requiem bell
Rang mournfully and clear.
For those who died that Easter-tide
In the springtime of the year.
The world did gaze with deep amaze
At those fearless ones, but few.
Who bore the fight so freedom’s light
Would shine through the foggy dew.

And back through the glen, I rode again
And my heart with grief shall soar.
For I parted then with my valiant friends
Whom I ne’er shall see no more.
And to and fro in my dreams I’ll go
And I’ll kneel and pray for you.
Though slavery’s fled, o’ glorious dead
When you fell in the foggy dew.

The Ballad of Ulak the Unconquered
Author Unknown

Bás Cliábhan
1106 C.E.

Godric stared at the blade, the Sword of Ragnuk, now the Sword of Gryffindor, forged from the form of pure war. It was every weapon ever created. He caught a glimpse of his reflection in the glittering, polished silver of the blade, which seemed to whisper to him, calling from beyond eternity, crying out for blood, for vengeance.

Ragnuk the Rampant had forged the sword for Godric, in exchange for a covenant between Wizards and Goblin-kind. Godric had done this without the consent of the other Founders, and Ragnuk had done this without the consent of all Goblins. Not that Godric necessarily needed the Founders’ approval; a representative of the little would always have a place at Hogwarts, covenant or no, this just formalized that understanding.

It heralded a new era of peace between these two peoples. Ragnuk the Rampant was the first in a long line of many goblin, or half-goblin teachers at Hogwarts, and his discipline was the fine art of Transfiguration, one of the most fiercely-guarded secrets of Goblin-kind.

It was a trade of necessity, of course. Goblins were not teachers, by nature. They were creators, artists. They had written volumes upon volumes, stored within the glittering vaults of Curd, Ackle, and beyond. They were not trained in the art of passing knowledge down from one living mind to another. Their speciality was in taking their knowledge and transforming it into something concrete and tangible, but dead.

Their mastery of artifice was an advantage; in some ways, they possessed some of the most eldritch powers of this new era of Magic. But their knowledge, which previously had been passed down from generation to generation in those tomes of lore, was rapidly decaying. They needed teachers to pass their secrets on, but more importantly, they needed to learn how to teach effectively.

Many goblins did not take kindly to Ragnuk’s perceived betrayal. Goblin secrets were for Goblin minds alone, they thought. They were not coins to be peddled, to be traded for carved sticks and silly incantations. But what choice did they have? Already, their charms had grown weaker, their famed prowess in battle had dampened. They needed to grow, to adapt.

And Godric, he needed a weapon, a weapon to teach him the ways of battle, the ways of bravery. He still shuddered when he dwelled on the memory of the Sword’s creation, that precious blade being forged and imbued with the essence of Void so as to take on the power of that which may harm it.

But what he remembered most was that phoenix, that precious phoenix, who had come to Ragnuk in his hour of bravery and need. He could see the pain and tears in the goblin’s beady, black eyes, and the shock in the phoenix’s final call as Ragnuk slid the blade through the fiery heart of the bird.

A heatless inferno washed over the room in an instant. The flame and the phoenix were both taken in by the blade, which glowed momentarily with an angry, ruddy light. Ragnuk held the blade in his hands and spoke.

“The blade chooses the wizard, Godric Gryffindor, not the other way around. Remember that, always. I pass this blade to you now, not in the manner of the trade of our kind. I truly give it to you, and you are its owner. It is a part of your heart, and you are a part of it. But just as you may one day give your heart to another, the blade may one day choose another as well, if need is great.”

And with that, in a burst of flame, the sword disappeared from Ragnuk’s hands and instantiated in Godric’s. Instictively, he gripped the hilt tightly, feeling the anger, the need of battle coursing through him. He could feel the finely-gilt writing that had been inlaid into the hilt of the sword, digging into his palms.

Nihil Supernum

There would be no blade that would ever come above this one. Only those of pure intent and noble heart would be able to wield it to its true potential. But such power comes with a price; it is lonely at the top, and if you find that you stumble, you will find that the rescuer hath no rescuer.

The memories washed over Godric as he stood floating above the great North Sea, staring at the triangular obelisk that jutted out of the waters, an unnatural blight on the otherwise rugged beauty of this place. This was an evil place, and within it dwelled an evil man; a dark wizard from origins unknown who was known only to the world by a series of epithets. The Grey Slayer, the Enemy, the Corruptor. He commanded a small legion of goblins who knew him as a-Jeroth, an ancient word that could roughly be translated as either “savior” or “destroyer”.

He knew that he was likely going to his death, like so many battles before this one. He waited, and waited, and waited. He knew, like those other battles, he would be waiting in vain, but he waited nonetheless, standing there, agonizing, over the call…

That wouldn’t…

Come.

His eyes flicked once to the stars above, wishing, hoping that just a single one would flash in the night. But it did not. He sighed, and steeled himself for battle. He drew the sword across his forearms, drawing blood in an ancient ritual: blood for blood. The blade took in the precious liquid, and the wounds healed themselves, but he could still feel their sting and he allowed it to fuel his anger, a reminder that death was close at hand.

He hurtled up, to the top of the tower, where six goblin guards stood watching dutifully. It saddened him to see them corrupted so. They looked among each other. They were ready. They had been expecting him. The one who was dressed in the most ornate armor and carried the most elaborate weapons spoke to the others.

“Ef yn dod, mae’r grissa ost drauka. Yw ef yma, yr un sy’n proffwydoliaeth dweud ewyllys i ddod â’r cleddyf y ffurlfen gwir rhyfel.”

The others nodded. Godric landed across from them, several yards away. The leader of the goblins then addressed Godric, in a broken, stilted form of the common vernacular. “You have come. It was said you would come. And it was said you have the choice, that you can leave now, before you bring death upon the world.”

“The only death I will bring today is upon your master, and you, if you continue to serve him. I am a friend of your people. Lay down your weapons and I shall let you leave in peace.”

The goblin spat upon the rough-hewn stone floor of the tower. “You carry the Sword of the Betrayer, he who sold our secrets to you, who would use them to tear apart this land, our world, even the stars in the sky.”

The light drizzle of rain had grown into a full-fledged storm by this point. The remainder shifted uncomfortably. They were unsure of themselves, their mission. Godric could see it clearly, that they did not truly believe in the cause. He had their attention, it was time to use their uncertainty to his advantage.

“I give you one final chance. Your minds have been twisted by your dark master. He has fed you lies, warped your thoughts. Leave now, or you shall die!”

He held his sword menacingly in the air, and nature itself seemed to respond to his battlecry: lightning crackled above, illuminating Godric’s silhouette, striking the Sword of Gryffindor itself. The power of the sword shielded Godric and those around him, taking in the force of the bolt and using it to augment its own strength. It glowed a brilliant white against the dark backdrop of night.

The goblins eyes grew wide. Good. Press the advantage. He moved to speak, but the leader of the goblins whispered softly to the others, “Ti’n gweld? Mae ei ei farcio fan fellten…”

This was not the reaction he was expecting. They stiffened, eyes narrowed. They carried themselves with grim resolve now, as if they suddenly had been given a reason to fight, a very good reason. They shifted into battle formation, and the leader shouted at Godric, “You are Death, and we shall end you!”

They shouted war cries, and rushed forwards towards Godric. They were six, six magic-wielders against one. It should have been a death sentence, but Godric was aided by War itself.

Time is finite, and as such not every subject and discipline can be studied and mastered. The art of wielding several people’s magic against one happens to be one of those disciplines. Such circumstances simply do not come up in the normal course of combat between magic users, and if it does, the situation simply takes care of itself without the need for special planning. One wizard simply cannot stand against the force of several combined.

There are rare occasions when one wizard is of such superlative power that they may stand a chance, but who could teach and train such a wizard on such circumstances? How could such a curriculum even be devised?

The result was slightly disorienting for the attackers, like playing chess against an opponent who simply does not move his pieces. Each attacker was expecting an individual, discrete response to their attack. But that is not what they were met with: his defenses were perfectly crafted to ward all of their assaults with brutal efficiency. The sword whispered hints, suggestions, and identified openings and weaknesses to be exploited.

He did not seek to wound or disable. These were servants of Death, and they had cast their lot, so he would send them to their master. They fought with a similar ferocity, for this man was the bringer of Death, and they would not allow him to bring death to their people.

It was a fight to the end, and despite being hopelessly outnumbered, Godric had them hopelessly outmatched. One of the goblins extended his arms a few inches too far when casting a curse, and the sword saw the opening. It prompted Godric to spin to avoid the bolt of light. With his left hand, he cast his wand in a fan-shaped motion to block the incoming elemental forces that were hurled his way from the blind side, and used the momentum of the spin to slice the goblin’s wand in two with the sword.

The goblin stood dumbly for a split second, mouth open, and Godric unleashed a kick which not only collapsed the lungs of the small creature, but sent him flying backward into one of his comrades behind him, who faltered. Another opportunity. Godric leapt into the air with preternatural strength, summoning wind and fire to turn aside both the physical projectiles and the gusts of ice that were directed towards him. He flipped forward in midair to dodge a series of spell-bolts, and then drove his sword downward through the top of the Goblin’s skull, all the way down until it reached the shoulder blade.

Ruined bone and brain splattered across the floor, and without sparing a moment, Godric wrenched the blade sideways, sending through it a flow of magic which caressed the dead bone of the goblin’s shoulder blade and arms, contorting them into sharpened spikes.

These were battle-hardened warriors, but even they did not expect the gruesomeness of their fallen comrade’s bones being sharpened into weapons. That surprise was the end of two more of them, as the spikes jutted outward and found their marks. The goblin leader, and one other were the only two that remained standing. There was a break in the battle.

Godric panted heavily, “Leave this place, now. Or you will die, like them.”

Ulak was the leader of the goblins, he had a wife at home, and a trio of younglings. It was for them he fought, and his eyes grew wet with the thought of them growing up without a father, of her without a husband. He wondered if they would know how he died, what he died fighting for, and whether they would continue the fight. He wondered if this man would rewrite history, turning Ulak into some callow villain.

He continued to fight, despite the rising hopelessness of the situation, breathing hard as he saw his final comrade blasted off the side of the tower with a concentrated burst of wind. With a quick glance, he saw the light had already left the goblin’s eyes before he even reached the precipice.

Although there were many more levels to the tower, levels and levels, they were the first and truly last line of defense. For if an attacker could breach their line, he could surely deal with the warriors inside. The scale of the tower was misleading; as large as it was, an outsider might think it host to an entire army, but the truth was that much of the place was unused. Ulak could not imagine a world with so many magical creatures and beings that this place could be filled to its full potential. His master apparently did.

He was staring Death in the face. Between the crackling of lightning, the downpour of rain and the crashing of the waves, it was likely that no one in the immediate lower levels heard the melee above. They had, of course, sounded the alarm from the first moment they saw the intruder, but since no reinforcements arrived from below, Ulak was certain the man had sabotaged their systems. He was not sure how, but the facts were clear. He was fighting alone.

Goblin honor dictated that he stand and fight, even if it meant his death. But what would he be dying for? There were those in the levels beneath him that would likely be slaughtered as well. And he could not risk the Gateway being lost. It occurred to Ulak that true bravery was not blindly adhering to a code laid down by those before you, that true bravery was making your own choice, even when that choice seemed impossible. Ulak would rather die than dishonor his name and the name of Goblin.

But then, some things are worth dying for.

He would not let death extend its reach any further than it had to. He would end the fight on his own terms. In one swift movement, he tore the metal ring from his belt and hurled it into the air. Godric’s wand immediately pointed towards it, tracing its flight path, but the ring expanded to several times its own width, and with a bright flash of orange light, it encased Ulak, freezing him in time.

Godric watched, his wand still following the ring, sword ready to strike, but the deed was already done. The ring, now a hoop, clattered to the ground with a loud CLANG, and Godric was alone.

Ulak, for his part, was gone. Gone, but unconquered.

On the bottom floor of the tower, Lord Foul stood, watching, waiting. He was wearing a jet-black robe, as befitting his moniker, and his face was obscured entirely by a billowing hood. As expected, the archway on the dais in the center of the amphitheater began to glow with an intense blue light, and the tattered black veil billowed violently, as if caught by some unseen gale.

This was a triumph… he thought to himself. He made a mental note, and continued to wait. He heard the clash of battle above him, the unmistakeable ringing of steel and crashes of magic and shouts of rage and cries of death. Patiently, he stood, listening, biding his time, until he could hear the footsteps clamoring down the spiral staircase near the back of the chamber.

There were a thousand and one ways that Lord Foul could have ended Godric Gryffindor long before this point, long before this battle had begun, long before Hogwarts was even founded. But now was not the time for Godric to die. No, now was the time for Godric to learn. It was time for the theatrics.

Lord Foul slowly clapped his hands as Godric made his way into the chamber.

“Congratulations, Godric, on making it this far.”

“Lord Foul,” Godric hissed.

“Lord Gryffindor,” he replied, mockingly. “I suppose you will want to have your climactic battle to the death with me here momentarily, end my reign of terror. Yes, yes, all in good time. But for now, come, come. I would like to show you something.”

Lord Foul gestured to a meter-tall lens, embellished with a fine platinum rim, affixed by an axis to a stand that looked to be carved out of an iridescent green stone. The stand could rotate, and the angle of the lens could be adjusted. The dark wizard spun the lens around to point at Godric, who could now see within its depths. He saw fire, all-consuming fire, a thousand phoenixes emerging from the conflagration.

“Do you see the phoenixes, Godric? Each one represents a choice, the choice that to this day, you have not made. Your phoenix will one day come, when you are faced with a moment that requires true bravery.”

“What do you know about bravery?” Godric spat. “You hide in this palace, you enslave a lesser people to do your bidding, you unleash your devastation from a distance. You are a coward.”

“‘Lesser people’? My, my, how high-minded of you. Are you not, as you say, ‘a friend’ to them? And yet you think of them so lowly…”

Godric grunted. “Enough word games. It is time for you to die.”

“No, it is not. I have a message for you. It is not a request. I will bring war and death to Hogwarts, for you cannot be allowed to persist in what you do. You know why you cannot be allowed to persist. You have seen it first-hand. Through this lens, I can see into your very soul.”

“Enough!” Godric drew his sword.

“Tell me, then, and speak truly, for I shall know if you are lying. You know the risks, you have seen them, you know what will happens if you continue, do you not?” Godric did not answer, he simply advanced upon Lord Foul. “Tut, tut. And you call me a coward.”

“I do call you a coward! Now show me your face and fight me!”

“And yet, you are too afraid to look inward, to embrace what you know to be true.”

“What I know is that I am doing what I must. Prophecy demands it. We must build a foundation for magic to be restored. Merlin, in his wisdom, put us in chains, because we were not ready for true power. We are teaching that responsibility, we are passing on–”

“Do not lie to me, or to yourself. You have seen the way your young wizards abuse their powers, ignore even the most basic and sensical of precautions. You teach Transfiguration to wizards who are barely of age! Already there have been accidents, already you have flirted with–”

“And what would you have us do? Stagnate? Rot? That is the world you envision, Lord Foul, a world of ruin and a world of death. That is not the world I choose to embrace, that is a world I will do everything in my power to prevent from coming to be. Yes, there are risks, but there is no risk greater than–”

“Than what? A world without magic? A world that is safe? For all of your supposed acceptance of what you call ‘Muggles’, you seem to view them much the same as you view the Goblins. Lesser. Impotent. Have you seen the wonders that they have created? You know the prophecies, you know that one day, they will reach the very stars, and they will do so without the touch, the taint of magic.”

Godric roared in anger, “They will reach the very stars in heaven so that they may tear them apart! Look at them! How they multiply, how they spread! Magic must rise, we must first raise ourselves up so that we may then raise them. Otherwise, they shall be the end of us all.”

“You speak of the prophecy, the one that goes by a thousand names. I wonder, how much do you truly know?”

“I know enough.” Godric took another step closer. They were within arms reach of each other.

“Do you? Do you truly?” There was a pause. “You know, we are more alike than you’d like to think.”

At this, Godric laughed. “Do you think you are the first dark wizard who has tried to tempt me with that speech?”

Lord Foul smiled. “No, but I am the first wizard who will show you,” And like a flash of lightning, his arm lashed out, grasping Godric’s shoulder, and he whispered a word.

Godric stood, reeling. “Ba. Egeustimentis Ba. Ba.”

“I have done nothing to alter your mind, Godric.”

“BA! BA!” He yelled, futilely.

“Say it all you like. I merely revealed new information to you. No magic can undo that.”

“Why? Why, damnit?” Godric shook his head, angrily. “What then? Why is this,” he gestured at Lord Foul’s cloak and around at the evil chamber, “Why is this the answer? How?”

“All in good time, Godric. You have served your purpose in coming here, child, and now I will take my leave.” Lord Foul turned and began to walk towards the archway.

“No!” He shouted, passionately. Lord Foul stopped, a step from the archway, and turned around. “You must tell me. How, how are we to stop this?”

“That is the riddle, isn’t it? You’ll have a choice soon. Very soon.” And with that, he removed his hood.

Godric’s eyes grew wide, and he staggered backwards. “You.”

“Goodbye, for now, Godric.” And with that, Meldh stepped through the archway.

Godric stood alone.