\chapter{Put Your Little Hand in Mine}

\DatePlace{February 1, XX,XXX}

\lettrine{A}{t} the moment the world began to die, few people felt it.
\SmallVSpace
Even fewer knew they could do something about it.
\SmallVSpace
Even fewer still had the requisite skills or knowledge to act on that knowledge.
\SmallVSpace
And even fewer still had the technology that could harness those skills into something useful for this particular situation.
\SomeVSpace
And so there were twelve.
\simpleline
\DatePlace{Earlier}

It was the ultimate weapon.
\SomeVSpace
It would decisively, conclusively, and immediately end the war, that secret war that had been waged since time immemorial. The enemy would be irrevocably destroyed, defeated in detail, sacrificed to the cause of the righteous. Of course, there were the doomsayers, proclaiming that the hubris of the project would end us all. It was hard to claim John was hubristic, however, when he subjected his system to every conceivable iteration of failure testing.

They had identified thousands of possible failure points, and fixed them all. That left seven distinct failure modes, and although they were fundamentally impossible to avoid, John’s team was able to decrease their probability to roughly one-in-ten-trillion each.

It still wasn’t enough to feel safe. They developed fail safes and response protocols to the failure modes, and John had personally rehearsed them all. Hundreds of thousands of times.
\SomeVSpace
It still wasn’t enough.
\SomeVSpace
A 1:\num{e88} chance was impossibly small, but it was still possible. So there was always the ultimate fail safe. The Line. He rubbed his right forearm like a touchstone. An astute observer would note that the system actually had but a single point of failure, and that was John, but John had personally accounted for that, as well. That small portion of his free will was locked away, in a place he could only access if they had really and truly \emph{won.}
\SomeVSpace
For a brief, bemused moment, he thought that the only flaw in the system was that the activation sequence wasn’t something more dramatic. It should have been a massive switch, or an ominous button, or some incantation. But, as it were, it was relatively unceremonious. A few keystrokes, and it was done.
\simpleline
The world shuddered.
\simpleline
\emph{No need to panic}, he thought himself as he went through the motions. He had literally rehearsed this exact scenario at least fifteen hundred times, enough to where the movements were rote. His team controlled the outputs and inputs and monitored the status of the buffer. John did the intense series of on-the-fly calculations in order to determine the precise initial vector, and after a few tense moments, the variables checked out, and he rotated the dial.

In short, they would simply roll the system back an hour, and start over. They’d have to triple-check everything. Twice. Each day. It would be at least another year before they’d be confident enough to try again. But, he’d waited this long. A year was trivial.

As the dial rotated past the origin, his forearm, which had begun to ache since the start of the process, now throbbed in earnest.
\simpleline
The world shuddered again.
\simpleline
At this point, it was cacophony. The team was visibly agitated. Some were even panicked. This didn’t make sense.
\SmallVSpace
One in ten-trillion is tremendously unlucky. But one-in-ten-trillion, squared? Probability analysis goes out the window.\\The question itself changes. It’s no longer, \emph{Is this just coincidence?} No, the question on everyone’s mind was simply, \emph{What the hell is going on?}  The possibility space was endless, but one immediately leaped to mind: \emph{Sabotage.} No matter, he couldn’t spare the thought. He needed to focus. They had still rehearsed the failure modes. This was still comfortably in the realm of their practice.  But the response protocol was drastic enough that everyone was agitated and on edge.
\SmallVSpace
“The lines, sir. They’ll be\mbox{---}”

“Short circuit the whole fucking physical system if you have to. The whole thing is fucked anyway! DO IT!”

“Sir, this is going to be a destructive read. If we can’t\mbox{---}”

“There’s no other options. Back them. Back them all up.”

“If it doesn’t work, we’re all{\el}”
\SomeVSpace
A pause.
\SomeVSpace
“This is a direct order.”
\simpleline
The world shuddered as the transmigration began.
\simpleline
John looked around. His colleagues were the first to go, their brains literally vanishing from their skulls, then converted into raw data, then pumped back into the system via the γ-class L.E. lines. As he scanned this displays, he saw the same scene playing out across the entirety of the system. If you zoomed out far enough, it didn’t even seem like much had changed.\\No explosions, no catastrophic crashes, nothing of that sort. After all most of the systems with the potential for catastrophic disaster were managed by the deadminds.

But seeing it up close and in person? He had seen people die before, very rarely. Usually it was willingly, people who had simply grown tired and were ready to “move on”. Idiots. They died as they deserved, peacefully but without pomp or fanfare. But these people did not will it, and they did not die peacefully. It reminded him of a puppet whose strings were cut. All the motive power that was keeping the awkward automatons of flesh balanced, gone in an instant. Billions, all massacred in the span of a moment.
\SmallVSpace
John made the snap decision to pipe the data from the payload back into the system as it was being constructed. If another component failed, he couldn’t risk losing all the data. It would divert a small measure of resources, and they would still have the physical storage structure to recover the payload.  The only downside was the potential for signal degradation; it was almost guaranteed that they would lose a few to noise, which would be a tragedy, for certain. But it paled in comparison to the possibility of losing everything. Besides, you could still recreate them, for the most part. The memories might be tricky to reconstruct, but at least they’d still be there.
\SmallVSpace
\emph{No.}\par
\SmallVSpace
Yes, there was noise. There was too much noise. Every signal was being garbled. Warped beyond recognition. There was interference coming from{\el} Somewhere? Only about 1,000 identities were piped through, and of those, the only thing left was raw DNA.  Change of plans.
\SomeVSpace
The payload was already constructed. It existed conceptually, in abstract. Now he needed to realize it. The Line was the most secure object in the known universe, and it had more than enough capacity within its buffer. He did more calculations. It would cut into its capabilities significantly. Maybe six hours, tops? It didn’t matter.

He’d saved the people. He didn’t save the world. The world was done for, but a world could be recreated easily. No. Not easily, of course. Nothing would be easy at this point. The system had failed at three separate junctures. This was not chance. Something, someone, was responsible.
\SmallVSpace
And that’s when he saw him. The man who was out of place, out of time.
\SomeVSpace
He was old.
\SomeVSpace
Old.
\SomeVSpace
No one was old anymore.
\SomeVSpace
This was his doing. There was no question. In pure reflex, he activated his Battle forms. He had even practiced this, fighting against countless unseen enemies. But, what good would it do? What to do? Fight or flight? What would he be fighting? What was the man doing? Those hand gestures were ancient. A past architect? A back door? No, the system was sacrosanct. Besides, the man had a tool. It was\mbox{---}
\SmallVSpace
The old man was holding The Line.
\SmallVSpace
No, No, No, no, NO.
\SomeVSpace
Flight. It was done. There would be no climactic fight to save the world or its people.  It didn’t matter what the old man’s motivations were, how he got there, anything.  Any time spent thinking about it was time wasted. There was no option left but to run, and to rebuild. He’d have to destroy the entire system, every last remnant though, to fully rebuild. He began to\mbox{---}
\SmallVSpace
No. No time.
\SmallVSpace
He didn’t have time. He’d have to do that part later. He’d have time later, but not now. It was time to run. He didn’t know what the old man was capable of, and none of this was rehearsed. He committed to the decision, and it was done. It was out of his hands now, so he had time to think, wonder, and speculate.
\SmallVSpace
\emph{Who was the old man? How did he get a copy of The Line? Is it even a copy? How will I recover the payload? How much of the system would survive? How useful would it be? How will I destroy it? What would this new world look like?}
\SmallVSpace
Questions, questions, questions. All the answers would be there, eventually.
\SmallVSpace
The system was procedurally generating humans as fast as it could churn them out. It started with the thousand or so genetic patterns it had recovered from the first aborted payload attempt. The rest, it built from patterns. Ten million and change.
\SmallVSpace
And then with all the fury of an exploding star, a new world was born.
\SomeVSpace
John emerged on the back end of eternity.