\chapter{Ordinary World}

\DatePlace{Hogwarts Castle, 3 Months Earlier
June 2, 1333, C.E.}

It was the Kiss heard around the world. The Kiss that launched a thousand conversations in the common rooms of Hogwarts. The Kiss that had the wise professors nodding, having seen the pattern before. The Kiss that had puerile young wizards practically falling over themselves to get the details.

It started, as most such Kisses do, with alcohol. Lots of it. In this case, several bottles of illicitly obtained Armagnac. Someone had nicked them from the Headmaster’s private store. It was their fifth year and they had just received the results of their O.W.L.s, and even in the 15th century, students were prone to celebrating bare mediocrity.

“Partying” had never come naturally to Nell. In her mind, celebrations were for events worth celebrating, and passing one’s Ordinary Wizarding Level test meant exactly that: you were ordinary. However, she had long ago learned that you simply cannot fix the world by yourself, for the most important part of the world is its people. And you can’t get people to change unless they like you.

Nell had also learned the necessity of altruism (if you could truly call it that). When you do right by someone, they want to do right by you, most of the time. There were always exceptions to every rule of thumb, but if she performed ten random acts of kindness, eight or nine of them would be returned in kind, which was more than sufficient. Nell rarely had only a single iron in the fire.

Her mother had taught her the value of being useful. If you constantly did the right thing by the right person, they would become reliant upon you. “Unconditional support gives you the ultimate power over a person, for you can withdraw your aid at any time, free of any repercussion. You must never ask for something in return,” she had said, “For this is no trade, and you are no merchant.”

She held sway over at least half of Hogwarts, students and professors alike. She helped Gryffindors with their homework and never charged a Knut. She assisted Slytherins in their small plots, and never called in a favor. She worked hard alongside Hufflepuffs and studied hard alongside Ravenclaws. She aided professors by corralling unruly students, grading exams, processing paperwork, and never expected any special treatment in return.

It was said that Perenelle was ambitious. Those of Hogwarts observed that three times she was presented with badges of honor: Prefect, Head Girl, and a medal of Special Service, and thrice did she refuse them. Was that ambition?

Hogwarts was simply made a better place by the presence of Perenelle du Marais.

Her power was unspoken, never once had she held her favors over someone else’s head or threatened the withdrawal of her assistance. For sure, there were those who tried to take advantage of her kindness, but she dealt with them easily: she simply was no longer kind to them. As such, she enjoyed a level of freedom in Hogwarts that few students had before and few would ever have again.

She walked openly in the restricted section of the library, she inquired about deep magics and high ritual above suspicion. When the school learned of the death of her parents, students and professors fell over themselves to offer her compassion, condolences, and charity.

It was because of this that, despite not hailing from a wealthy family, Perenelle was able to afford a trip across the Old World during the summer of her fifth year. It was because of this that, despite not hailing from an ancient family, she was practically handed a roadmap of secrets that guided her travels, ensuring she would return from the journey enriched with lore.
\simpleline
It was also because of this that Nell permitted herself a celebration, and it was also because of this that she found herself in an uncomfortably small cabinet with Festivus Weasley, waiting patiently for Headmaster Gagwilde to depart for dinner so that they could pillage his unnecessarily large collection of unnecessarily expensive spirits.

“You appreciate the fine arts, right?” he whispered. She rolled her eyes. She wasn’t sure where he was going with this, but it was sure to be cringe-inducing. She didn’t respond.

“I’ll take that as a yes. You know, this is usually the part of the play where the wacky, dashing hero and the beautiful but shrewish heroine get pushed into each other’s arms by some improbably ridiculous combination of accidents and physical comedy. And it usually ends with a kiss.” He coughed. “Hint, hint.”

“Hint, hint: when you’re flirting with a girl, it rarely pays off to call them a ‘shrew’ in the very same sentence.”

“You wound me, dear Nell. The shrew in this situation is none other than myself. In a delightfully subversive twist, I am the beautiful heroine of our own little comedy. You, my dear, are the persistent hero that simply can’t take ‘No’ for an answer. Although… If you were to ask me out now, who knows if my answer would change!”

“Oh, I think I’m fine not knowing the answer.”

“Some Ravenclaw YOU are, ignoring a riddle like that.”

“Some Gryffindor YOU are. You haven’t once directly asked me out without hiding behind a joke.” If there were space, this would have been where Nell indignantly put her fists on her hips and looked imperiously up at that oversized, fire-headed twit. But as it was, they both were awkwardly stooped over and no such dramatics were possible.

“Will you go out with me?”

She stifled a laugh. He elbowed her ribs. “Oh god, no! Of course not!” She stifled another laugh. “I’m not even going to bother with some silly cliche like, ‘I don’t want to ruin our friendship.’ No. Just no. A thousand times, no.”

“You’re a devil-woman, you know that? This summer, I’m going to find myself a nice Veela, and then you’ll see what you’ve been missing out on!”

“So you’re saying you want me to watch? Gross. Also, no.”

“Oh. No. Nothing so crude. Our lust will be so all-consuming that we can’t help but fly into fits of passion everywhere we go. The Great Hall, the classrooms, the hallways, your desk… It’s just a statistical inevitability that one day you’ll be minding your own business, probably doing something Ravenclaw-ish like reading while walking, and stumble upon us.”

“It’ll be easy enough to avoid, I’ll just steer clear of any unpleasant smells. It’s already nearly unbearable in this cabinet, I can’t imagine what horrific scents would emerge from you if you were to sweat. Now, shut up. The Headmaster is leaving.”

She had made sure to cast an unnecessary strong Silencing Charm earlier; she knew how Festivus loved his banter. They watched through the crack in the cabinet as the Headmaster gathered his things and departed. They emerged, looked around, and began scanning the office. A portrait on the wall coughed loudly.

The noise came from a portrait of an old, wizened Mage with a mischievous grin on his face. He was nonchalantly looking another direction, while clearly pointing at a bookshelf. Nell winked at him. It was the portrait of old Headmaster Porpentine, for whom she had arranged an illicit Portrait Passage years earlier, giving him direct access to The Bawdy Brothel of Bathsheba, a famously explicit painting by Lord Dolomphius LeValley. As they walked over to the shelf, the portrait coughed again, “Prometheus Bound.”

Fortunately, Nell was fluent in several dialects of Greek, both ancient and modern, and recognized the book. It was ancient. Or at least, it looked ancient. Did they have “first editions” in Ancient Greece? She reached for it, pulled it slightly, and realized it was on a pivot-and-latch mechanism. As the latch came loose, the case swung on a hinge and opened to a secret passage whose walls were lined with hundreds of bottles of wine, spirits, and ales. They quickly loaded up Nell’s mokeskin pouch, rearranged the bottles to make it less noticeable, reset the trap door, and made their way to the exit.

“Thanks! And by the way, we were never here!” She whispered to the portrait of Headmaster Porpentine, but he had already disappeared. Through the gaps in the Portrait Passage, she could hear the faint tinkle of amorous giggling. She grinned and rolled her eyes.
\simpleline
Nell was famous for her self-control, even when she had consumed more than a few drinks. And she had consumed more than a few drinks that night. But when you are so close with someone for so many years, you begin to notice the subtle signs, like a rope becoming slightly frayed around the edges. And Helena Ravenclaw and Perenelle du Marais were very close, indeed.

Ever since they were first-years, they bonded over shared interests, personality traits, and philosophies towards life. They were both devastatingly intelligent young women in a world that did not look kindly upon women doing anything beyond bearing children and tending shop. For certain, there were the titans of old, the Helga Hufflepuffs and Rowena Ravenclaws and Galath Ollivanders. But for a young woman to aspire to such lofty heights was looked at with the same condescending smiles and nods that a wizard might give a young boy who says, “I want to grow up to be like Merlin!”

Further, they both were fiercely competitive, both with each other and the outside world, and they both hated to lose. Nell had never quite learned how to lose, and Helena rarely had cause to. And perhaps most importantly, they both wanted nothing more than to be recognized for their skills and talents, rather than their undeserving gifts of genetics and lineage. Even as a young girl, Perenelle was captivatingly beautiful. It led to quite a lot of unwanted attention from unsavory people, and the old nursery rhyme her father had taught her still echoed in her mind:

If there is a doubt
Just raise your hands and shout!
Those silly acrohandulas
will run away and pout!

Nell did not want to simply be the dumb, pretty girl. Her parents raised her better than that. She held herself to a higher standard. It made her work even harder to prove that she was more than just a porcelain face, piercing eyes, and ample bosoms. Not that it did her much good. She was careful, though, not to ignore her gifts either, as they opened doors that would otherwise have remained closed more often than not.

And there was Helena Ravenclaw. The final remaining name-descendent of the Ravenclaws, and the final remaining name-descendent of any of the Founders. The bloodline was still alive and well, of course, but there was power in a name, and given that she was an only child, she was the death of the Ravenclaw name. Everywhere she went, she carried with her the unwanted aura of history, and the air was heavy with expectation. She desperately wanted to be known for being something other than The Last Scion.

They both were secretly terrified of being a footnote in the grand tale of their companion. Helena, the Dorky Friend of that Hot Ravenclaw Witch Who Basically Owned Hogwarts. And Nell, the Insignificant Sidekick of the Titan of History and Prophecy. They both knew their own fears, and as such, knew the fears of the other. It went unspoken yet understood, as did many things between them, which only strengthened the bond of their love and friendship.

Helena had more raw talent than Nell, but Nell was more cunning and more familiar with the more obscure (and thus powerful) spells and rituals. Helena knew the intricacies of Magic as intimately as Nell knew the intricacies of people, and together they made a formidable team.

Nell did have one crucial advantage: she had a much greater capacity for alcohol, which was fortunate because she had consumed quite a good deal of it this evening. That capacity was quite apparent, especially because she had volunteered to be the test subject of Festivus’ new ritual. When he explained it to her in the Common Room, she quizzically cocked an eyebrow and asked, “So, if this works, then what was the point of our escapade in Headmaster Gagwilde’s office?”

“Isn’t it obvious? It gave me the perfect opportunity to ask you out!”

“And how did that work out for you?”

“Swimmingly, if I do say so myself. With every loss comes opportunity: Porpentine is a dirty old bugger, and his portrait told me about the secret peephole into The Bawdy Brothel.”

“Gross. Now, what if this doesn’t work?”

“Well, it could turn that water into anything from a love potion to a Draught of Living Death.”

She shrugged. “Great! Let’s give it a shot.”

The room grew silent as Festivus drew his wand. Always the dramatist, he let the anticipation build. And build. And build.

And build.

After an obnoxious amount of silence, he lifted his wanted, and the crowd swelled with expectation.

He let them wait.

They groaned loudly. Someone chucked a Pumpkin Pastie at his head, which he deftly caught with one hand. He took a bite, chewed slowly, delicately wiped his lips, and finally, began the incantation:

Pesternomi Peskipiksi
Turn this water into whiskey!

Silence. There was no discernible change in the cup that Nell was holding. But that was not indicative of failure. A skilled dramatist herself, she held up the glass, gave it a sniff, and paused pointedly.

The crowd pressed inward, trying to get a closer look, hoping to catch a whiff. As if in response, in one swift motion, she lifted the glass and drank the entire thing in one gulp.

The room was silent. The anticipation was unbearable.

Then Nell made The Face.

The room erupted into cheers. Men hugged, women swooned, and for a brief moment, Festivus was king. Someone had hastily assembled a fountain in the middle of the room, and Festivus went to work on casting the ritual again. The young witches and wizards flocked, with goblets in hand, to the fountain which now sprayed forth voluminous jets of clear spirits

Nell, despite herself, was impressed. It was a sacrificial ritual which delivered unto the caster a fixed quantity of alcohol at the expense of an equal quantity of water. It was barely 16 syllables long, invented and cast by a student who was barely 16 years old. That was impressive even by her standards.

Centuries later she would look back at this moment in a much different light. The amount of energy in that sacrificed water could have leveled Hogwarts 1000 times over. In the days of Grindelwald’s reign of terror, she and Meldh had guided Muggle scientists with a hidden hand, helping them craft a terrible weapon which was a triumph thrice over: in one fell swoop, it had destroyed the collected lore of Terumoto and Sumitada, it had broken the will of Grindelwald’s allies in the Orient, and it created a tenuous balance of power in the Muggle world. “Mutually assured destruction” had ensured peace in the Wizarding world for centuries, and now the Muggles had that same protection.

This careless ritual was fifty times more powerful than that weapon. Such power in the hands of a boy who was not even a man. His wand, a devious facsimile of Gom’Jorbol’s original anchor, the Rod of Ànkyras, ensured that the energy was harnessed safely and efficiently. But the danger was still there and it was appalling.

At such a young age, Nell had no way of knowing all the secrets of Gom’Jorbol’s staves, so she was blissfully unaware of the full extent of the danger. Had a single Dragon heartstring laid out of warp with the Yew shell of his wand, that energy would have reflected back upon itself and vaporized the whole of Scotland.

In the present day, the end-times, Perenelle knew that she was far too valuable to risk such possibilities. Perenelle knew now the true danger, and she knew now the price that the multiverse would pay for her failure.

But centuries earlier, she was simply a teenage witch, impressed and more-than-slightly drunk. Centuries earlier, her response was the face. Helena, for her part, knew that Nell was acting for the benefit of the crowd, trying to make the party that much more memorable. Nell never made The Face, even when she had consumed much larger quantities of booze at one time than she had just now.

But, Helena also knew that Nell was not unaffected by the drink. Her normally sure stance was just a hint more wobbly than usual, her typically crisp diction slightly less precise, her keen, sharp eyes a fraction less focused than normal.

Helena knew the signs, and knew the effects, and she figured, what better time than now? “Nell! Are you excited for your trip?”

Nell smiled when she spotted Helena. “Yes, oh yes. Professor Ollerton has given me some great leads, as have the Nutcombe hags. I have enough money to make it all the way to Greece, and if I’m lucky I made even be able to visit Arabia.”

‘You know that if you need anything… You know, Galleons–”

“No. Helena, no. I wouldn’t ask that of you. I don’t want you to feel…. I don’t know. If something were to happen, I don’t want you thinking that you were responsible for it.”

“You know that I could never NOT feel that way. If I ask you not to go, if I told you our friendship depended on it, would you still leave?”

Nell paused. Was this her way of asking? Or was this simply hypothetical? “But we both know that you would never ask that of me, we both know that you would never make such an ultimatum.”

“I know. And believe me, I wouldn’t do that to you. I’m just saying, what if I did? Would you still go?”

Nell paused, again. No, Helena wasn’t asking. And for that, she deserved honesty. “No. I wouldn’t go.”

“So. In a way, I do have the power to stop you and I’m choosing not to. So if something were to happen to you it would, in a way, be my fault.”

“You don’t need to worry. Seriously. I can take care of myself.”

“I know. But I’m not the one who brought up the danger, you are.”

Nell sighed. “I guess you’re right. I’m going to some dark places, and I will probably meet some dark people. I guess if I’m being honest I’m a little bit scared.” She did not let on just how dark were the places she was visiting or unscrupulous were the people she was seeking. She did not let on how scared she was.

Helena’s heart was racing. Here was her chance.

She took Nell’s hand. “No, you’ll be fine. We both know you will.”

“Yeah. You’re right. Well, umm.. I guess I should, I don’t know. I guess I should say, goodbye.” Her eyes were glossy, betraying the tears she had successfully fought back. Nell’s tone and expression were somewhere between “Goodbye, see you in the fall,” and “Goodbye forever”.

And in that moment of recklessness, Helena pulled Nell close and pressed their lips together.

What.

If there is a doubt just raise your hands and shout no we shouldn’t do this Yes why not she wants it so much she will owe you forever she wants it so much you can use this No friends don’t use friends Stupid silly ignorant of course they do friends use each other and make them feel better while doing it No doesn’t feel right Yes it does you know it does you have wanted this we know we have wanted this to see to look to feel to taste not seriously not for real just a taste yes just to taste you could have just a taste think about what you want what’s the harm no one gets hurt everyone wins everyone wins you’ll be doing the right thing to do the right thing she is broken fix her fix her fix her fix her fix her FIX HER

Nell gave in.

In the background, Nell could hear the bawdy cheers and hoots of the other students. Witches did this sort of thing all the time for attention, so no one thought much of it beyond a moment of alcohol-fueled experimentation. Despite that, The Kiss was all anyone would talk about for the next few days, the rumors made all the more lascivious by the fact that the two had disappeared from the common room, not to be seen again until the next day.

Unlike the rest of Hogwarts, Nell and Helena would never get the chance to discuss The Kiss ad nauseum. When Helena woke up the next morning with a pounding headache, dry mouth, and bleary eyes, Nell was already gone


