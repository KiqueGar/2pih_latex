\chapter{The Transmigration}
\DatePlace{Igbo-Ukwu\\
Present Day}

\lettrine{C}{hinwendu} and Nnamdi strode hand in hand through the ruins, holding spears in their free hands, which doubled as walking sticks. They walked for many hours, past the ruins, through the dead place. The farther they walked, the more death surrounded them. Dead grass gave way to loose scrabble and dirt, and when they at least reached their destination, Chinwendu stopped.

“We have come to this place, Nnamdi, so you may finally become a man, and hear the story of our people, that only our men may hear.”

“Why is it that only the men may hear this story?”

“The women tell their own tale, as do the birds and beasts. Such stories are not for your ears, nor is our story for them to hear. It is a tale of our people, where we came from, and where we shall go. It is a tale of our magic, how like the little weaver-bird who once flew to close to the sun, as did our people, all people. It is why we are colored the way we are, and why we must have pity and mercy on the foreign ghosts beyond our country: despite their evils, they were not graced with the presence of Anwu as we were.”

“But, how do we know that our tale is the true one, if we cannot listen to the tale of the women, or the tale of the beasts?”

“Ah, Nnamdi, you are not old like me. You have an entire world inside your head, you and all the children, when you play in the plains while the men and women work. You see the world in front of you, rigid with its rules and laws, you see the order and structure of our village and how it is not at all like the fantasies of your imagination.

“And so you grow to ignore the world and stories inside your mind. I have seen many things. I have heard many things. And if there is one bit of wisdom I may pass to you so you may pass to your sons, it is that all tales are true in their own way.

“Now sit, and you shall hear the tale my chief told me, and his chief told him before that, and his chief before that. It is a tale that stretches back uncountable seasons…”

The Palace at Arcadia
903 C.E.

THE FOUR SIDES OF THE SQUARE HAVE ENTERED THE BOARD
THEIR CRESTS SHALL BE THE BLOCKS BY WHICH MANKIND’S CUNNING SHALL WAX
THOSE WINGS WILL BEAR MAN TO THE STARS, BUT THEY AS ALL, MUST MELT
AND THE FALLEN HERO SHALL SWING HIS BLADES
AND IN RETURN THEY TOO SHALL SWING
BY THEIR ASHES THREE OF THE THREE THREE SHALL RISE
THE ENLIGHTENED TOWERS, THE HALLOWED GODDESSES, THE STOWN-HEWN SERPENTS
TOGETHER THEY MUST CHOOSE TO FOLLOW THE PHOENIX OR SOLVE THE RIDDLE

“This is… complex, to say the least,” Meldh finally spoke.

“Yes, but is a tapestry that has been woven ages ago, long before this day.” Merlin replied.

“Are you sure it is wise to keep so many threads in play as one time? The people may be ‘ants’, as you say, but I have learned that things do not always work out–”

Merlin cut him off with a gesture. “Perhaps for you. Step on an anthill, and the aftermath may be unpredictable, thousands upon thousands of them scurrying at random, barely comprehending the cataclysm that destroyed all they have built. But as surely as the sun rises, they rebuild and go on. By the time the next hill is built, the cataclysm has passed into ancient history.

“When I step on this hill, it will matter little. As you said, they are cunning. They will find a way to rebuild, as they must, and the manners in which they may do so are limited. And more to the point, all of which are to our advantage. The most obvious choice, and thus the inevitable one, is that they circumvent my Interdict by formalizing the passage of knowledge from one living mind to another.”

Meldh nodded. “A school.”

“You will either succeed in destroying that school, and we shall rebuild it again in our own image, or you will fail, and the world will unite as it never has before, in defense of a common foe and his allies. Either way, it is a victory.

“When the interdict comes to pass, it will be felt across the world. Those who did not bear witness to the event shall be confused, scared. And once they are made aware of the fact that they were not among the few who did bear witness, they will retreat, entrench their positions, resentfully hoard their lore. After a few centuries, those in the north will sense the opportunity, and embark on a great crusade to claim that lore from the hands of those with which they have split.

“As with all wars, the descendants of Atlantis will use mortal men as their pawns, and use the carnage to further their own ends. Long has the wizard-kind of this kingdom sought revenge on the Fae and the Goblins, and you shall give them further cause to do so. I suspect they shall enslave the Fae, bind them with their magic into servitude. And they shall oppress the Goblins, which will be quite convenient for us, as we can play the two against each other for centuries to come.”

Meldh considered this in silence for a moment. “And what of they holy relics? The Seljuks are in possession of Neirkalatia’s Cross, and they will not surrender it lightly.”

“The Cross shall be taken, I will see to that personally. Such artifacts of power do not fall into the hands of the masses. Its new owners will seek to protect it, shroud it in mystery. Like so many other cults of power, they will form a secret society to protect its lore, pass down its knowledge. They will shepherd it for us until Ragnarok.”

The Arch that stood behind them whispered softly, the brilliant white veil billowing slightly in the windless room. They were surrounded by marvels of gilt and glass, and dotting the room were various tables and plush chairs, constructed of the finest quality. The floor was a stone that took in the light with a soft quality such that it was not painful to look upon in full light despite being the purest of white.

Merlin began gesturing, and spoke as he did so. “I think a change of scenery is in order before we begin.”

With but a thought, their surroundings began to transform. Their sparkling glass paradise slowly melted into brilliant grey stone, and the various seating arrangements merged together into a raised amphitheater with Merlin at its center. Meldh dutifully took his place in the audience, and observed as Merlin himself changed as well. His young, brash, and beautiful form slowly melted into that of an old, wizened leader. He raised the Cup of Midnight.

When Merlin spoke, his voice was other-worldly and echoed within his mind rather than within the chamber. “Come, come, come, those of puissance, you Lords of those of flesh and blood, of all of nature’s creatures, touched by Magic. Come, come to me. It is I, Merlin, first among you, Prince of Enchanters. Come.”

He could have summoned the entire world if he had so chosen, but that would be unnecessary and foolish. A few select leaders of a few select regions would be more than sufficient to seed the legends. The true ritual would affect the entire world, regardless of those who were in attendance to witness.

Ignorance and mystery were their allies. The rulers who were not present would surely find out, and the ensuing conflict would be to their advantage. Those who were too remote to hear the news in any sort of timely fashion would create their own explanations and tales. They would remain shrouded in ignorance, their progress stymied by their lack of understanding.

Those affected by the Calling could hear the voice as clear as day, a harsh whisper from within their own minds, beckoning them: “Come, come, come to me in the seat of my power, for my days grow short.”

Merlin’s name alone was sufficient to command the audience of the most powerful wizards of the day. But even had they wanted to resist, they found themselves compelled by the inexorable pull of Merlin’s magic. One by one, they Apparated into Merlin’s tower, and the silence was punctuated by dull pop after dull pop. When the room was full, and Merlin was satisfied with the attendance, he began.

“I am old, my friends, as are many of you. So I will speak swiftly and to the point. Atlantis is gone, claimed by a horrible tragedy beyond reckoning and comprehension. It is sealed beyond time, and with it, its secrets, but most importantly, its protection. There was a time when all men had Magic as we do, and all men knew the dangers, all men knew the precautions necessary to protect themselves and the world.

“That time is no longer. Not one man in 10,000 now is a descendent of those noble people, and of those, they have not one piece in 10,000 the knowledge those people had. The world grows large, once again. And the days of one wizard ruling 10,000 men are gone. This growth, if allowed unchecked, will surely result in disaster.

“Imagine, the combined power of all in this room, multiplied by a hundred-fold, waging a great and terrible war against an equally sized force. It seems unimaginable, but within a few short centuries, that will be the reality. The world grows, and with it knowledge, and with that, threat. Magic is a great power, yes, but it is also a great responsibility, is it not?”

This remark drew grim nods from all those in attendance.

“You are not just the rulers of your lands, you are its shepherds, its stewards. Despite the cries of tyranny, despite the ungrateful accusations from the very people you protect, you stand true and noble. You give them life, you give them love, and even though they spit on your name, you allow them to grow and thrive.”

Scattered applause, a few cries of agreement.

“No man lives forever, but in spite of this, it is no secret that we live far longer than those touched by Magic. Some of you in this room several centuries old, and you who have watched the ebbs and flows of time have seen firsthand how the world has changed.

“Glewlwyd Gavaelvawr, there was once a time when your people were prosperous, and you were free to spend your days helping them build. But I have watched you over the years, and more and more of your days have been spent preparing for war and battle. You have been defending your lands rather than growing them. You withstood the barbarian hordes–”

Glewlwyd chimed in, “Yes, but that was nothing compared to the invasion of the Greeks. Of his people.” He pointed a finger at Meldh, who stood opposite Glewlwyd in the amphitheater. “You, Mundre, from the City on the River.”

“Do not pass the sins of the father down to the son. My ancestors and those who came before me may have brought war to you, but my people have recanted, we have relented, we have left you in peace, and we have opened our doors to you in the spirit of trade and prosperity.” Meldh spoke.

Merlin intervened. “And that, precisely, is the problem, my friends. Society is a fragile powder keg. It takes but one spark to ignite, to lead to war. And with our knowledge growing day by day, not only are there more potential for sparks, but the price of war becomes more and more untenable.

“I have many subjects about which I wish to speak to you, but we must respect the traditions of our kind, and so let us first begin the Ceremony of the Gifts.”

In keeping with the spirit of Noblesse Oblige, it has long been a tradition among wizardkind for the most powerful members of the community to bestow gifts and favors upon the lesser. And in that spirit of nobility, these gifts and favors were rarely for personal gain, but rather for the benefit of their subjects.

The leaders of Britain, Europe, Rome and Greece among others came forth with their requests. Advice on magical theory, assistance with enchantments that were outside their skill and knowledge, all manner of things that to Merlin were harmless bits of hedge magic and parlor tricks. The last of them, Glewlwyd Gavaelvawr, was accompanied by a chieftain from his lands of great import but very little power.

The chieftain spoke. “My lord Enchanter, Prince among Princes, this is a matter of which it embarrasses me to speak, but I must. My wife is the treasure of my heart, she has born my children and claimed my heart and soul. The thought of life with her is… The pain of those thoughts is too much to bear, much less if such a thing actually came to pass. There is a great Seer in our village, and he has foreseen that my wife will one day become the wife of Lord Edmond of the Noble House of Black. How might this fate be prevented, how might it be stopped? I know that such a request–”

“Fool!” Merlin exclaimed. “Although there are those who would argue otherwise, you should know that all prophecy is true in its own manner. We live in treacherous times. Time has but a single thread it may span: et quod dicitur erit quod. And if you differ by so much as a grain of sand, you risk a fate far worse than your wife one day marrying a noble Lord. Prophecy is not something the untrained should dwell upon.

“You may one day pass before her, would you wish her to be lonely? Perhaps her choice of husband after you were gone would be Lord Black, and perhaps he would ease the pain of your loss. Or, perhaps, in your single minded quest to avert prophecy, you neglect her and drive her into the arms of a lecherous Lord. Prophecy forms strange loops, and it is best not to entangle yourself within them. Heed the matter no more, Sir Davies, and your world shall be better for it.”

“Yes, my lord.”

“It is now on the subject of Prophecy that I wish to speak. You have all known for some time of one of the key prophecies, the once and future King who is marked by lightning and whose arrival is marked by thunder, he who shall pluck out the very eyes of heaven. This man may be our savior, or he may be our damnation, but we must not seek to delay or forestall his coming. We must simply prepare for it, and pray that when he arrives, he does not choose the path of Death for us all.”

“And how may we stop that end, how may we preserve the world of life?” Asked Meldh.

“So that we will not suffer the same fate as our forebears from Atlantis, I shall bind the world of life, and seal away the most dangerous and troublesome secrets of Magic. Knowledge spreads like a plague, and there is some knowledge that is not meant to be passed unchecked. I will ensure that the most powerful of Magicks may only be passed in their entirety from one living mind to another, to ensure a pure path of succession of such power.

“You in this room who have lived to see dynasties rise and fall know the implications of this. The most dangerous, the most powerful bits of Magic, the ones that give their wielders the most singular advantages, they will not be shared. They will be hoarded, and because no man is immortal, one day those secrets will die along with their owners.

“It is in this manner that only magic worth sharing, worth spreading, shall persist. Magic which makes life better for all rather than concentrate power in the hands of the few, that is the magic that will proliferate, that is the world we shall craft. The dark magic and eldritch rituals that are being discovered on a daily basis shall no longer freely flow to whichever lucky adventurer happens to stumble upon the grimoire of someone far more ancient and wise than himself.

“And yes, it is certain that there will be some who wish to keep these bits of Magic alive, and they will form orders and mysteries and cults designed to protect these secrets. You, who are the stewards of this world, must seek out these demons and purge them. This is my Interdict, and that is my mandate.

“But even this spellcraft, both blessing and curse it may be, shall not be enough. Man is a cunning creature, and even without the aid of Magic, his knowledge will grow, and one day their power shall surpass even ours.”

At this, the leaders of the Wizarding world looked at each other, and they were greatly troubled. Many murmured in disbelief.

“Prophecy has foretold this, that one day mankind will touch the stars, a power which is beyond even the greatest of you. But of that Prophecy, I shall speak no further. Instead, I tell you this.

“The Greeks came to our island as invaders, joining with the Faerie and the Goblins to lay waste to our places of power, as you too well remember. But Britain is a strong land, and we resisted them, showed them the rightness of our ways, and we have joined as one, combining our lore to do great things.

“There will be invaders in the future, but of a different sort. They will seek to bring the entire world to its knees, and with them they will bring fear and ruin. This is the Apocalypse of which I spoke. What we have here will not last, for no man is immortal. New orders will rise, and with them a new order shall arise. I have seen this, and now, I ask you to bear witness.”

As he spoke these final words, he overturned the Cup of Midnight, its effulgent, inky black Void flowing forth, blanketing the entire room, the entire world, for Life and Time. For an instant, an eternal instant, the world was dark. And in that darkness, a voice cried out. Not yet a man’s voice, but not a boy either. It spoke in hollow, clipped tones:

THE FIRES OF THE SOUL ARE GREAT
AND BURN AS BRIGHT AS THE STARS

Ha’Rova Ha’Yehudi
Moments after

Anka looked up at her mother. That was odd. The torch by which she was reading must have flickered. Or something. She blinked away the momentary darkness and looked back at the scroll. Her parents were scholars, so she was one of the few children her age who could read. She was browsing her mothers writing on the Ritual of Flight. Something about it though, something didn’t make sense. She had it moments before, but now… Once her concentration had broke, she couldn’t understand it.

She grasped the cursory incantation of Levitation, and the basic principles were the same. She read the words, and the theory should have made sense, but she just couldn’t make it work. She had seen her mother fly with her own eyes, so she knew it was possible.

“Mama,” she asked. “I don’t quite understand.”

Her mother stood up from her desk and walked over, putting a warming hand on her child’s back. “It’s quite simple, Anka dear. It’s the same premise as Wingardium Leviosa, but with a few simple tweaks. Here, let me show you.”

She walked her daughter through the bits of hand gestures and the proper frame of mind. Immediately, it clicked, and Anka understood. She performed the Ritual, and rose from her chair.

How very curious, indeed.

Zaqatala
Moments after

Georgi Abashvili was disturbed by the momentary darkness. His brother had long since taken leave of this life, but Georgi had persisted like a bad cough throughout the many years. He was old, and he could feel the ache and pain the world and in his bones. He felt that ache in a new way, now. Something was different in the rustling of the leaves, the soft gusts of wind, the way the light glittered off the Caucasus mountains in the distance. He was old, and his head was already stuffed too full of useless knowledge. Although he could not put a shape to the Interdict that lay on his mind, he could feel its presence, the same way he could always feel when someone else had sat in his favorite chair.

No matter. He had experienced upheaval before, he would experience it again. He took a long drink of goat’s milk, and wrinkled his nose slightly, for it had turned. He closed his eyes, and resumed his meditation.

The Headwaters of the Misqat’nk River, Nipmuc lands
Moments after

The guardians of the Sleeper waited, for that was their role. They waited in darkness, waiting until he who was marked by lightning would emerge forth from the Voice, and bade them wake his master. When that day would come, and not a moment earlier, they would open the sacred Scrolls, laid down by the Sleeper himself. They would read the Ritual of Awakening, learn its secrets, and call forth their master from his dreamless sleep in the City of the Dead.

Although none of them could feel it, none of them could sense it, somewhere in the forgotten soft places of the world, the Sleeper shifted in his rest, for he knew this day that he would never again wake.

Glen Nevis, Scotland
Years later

Ollivander, who now called herself Helga Hufflepuff, still reeled from Meldh’s betrayal. Her dream, her heart’s deepest desire was crushed. Meldh and his companion had lied — no. Not lied. They had told her what she wanted to hear, and that is what she heard. They said she could help with with her grand design.

She wanted to elevate all of humanity, not simply Wizards, but every last man, woman, and child, to gift them with the blood of Atlantis. She was no fool. She would implement safeguards, she would limit magic, not just for the newly ascended, but everyone. The Interdict of Ollivander, it would force magic to be channeled through a wand. Wizards across the whole of civilization were already well used to her devices, and through that she exhibited no small measure of control. She envisioned a world of wandholders, doing great, magnificent things, channeling their power through her creations.

But there were so many missing pieces, and she was not patient. When Meldh and his companion showed her the means by which she might accomplish her ends, she was blinded. She willingly relinquished control of her Cup to that man, the ruler of Magic-kind in this corner of the world. He was known for his wisdom and benevolence, and she trusted Meldh’s judgment.

She trusted him because they were bound by something far more deep than even an Unbreakable Vow: they were bound by the honor of their kind. The word of an Immortal is inviolate, it simply must be. No matter how long they may live, one simply cannot enact one’s grand plans without assistance. If their word were not their bond, what other coin could they spend? Threats are too often empty, bribes too often worthless. To violate the sanctity of one’s own word even once is to render one impotent: if you cannot be trusted, you have no allies.

So although they swore to help her, swore they would accomplish her ends, she should have listened more carefully to their well-chosen words, to the promise they both made. They did not say how, or even when they would grant the gift of Magic. They did not specify the means in which they would enforce her safeguards, only that they would see to it that the Interdict was put into place.

She knew now the true meaning of the prophecy, the one concerning the four sides of the square. She knew this was how she was to spread the Gift for the time being, and she knew that as long as Meldh was alive, there was hope.

And so it was that Helga Hufflepuff, her apprentice Godric Gryffindor, the bookkeep Rowena Ravenclaw, and the scholar Salazar Slytherin had banded together, the four pillars on which a new renaissance of Magical education would begin.

It took decades for them to plan, to execute. The sheer logistics of it were seemingly intractable at times. How would they inform the parents. How would they find the teachers? How would they pay the teachers? Who would write the curriculum? Where would the students stay and who would feed them?

Together, they mulled over these questions. Long nights stretched into bleary-eyed mornings which gave way into sleepy afternoons, all spent together, discussing, arguing. Oh, the arguing. Godric had grown now, and Ollivander in her new identity as Helga Hufflepuff had allowed herself to age as well. The sting of Meldh’s betrayal was still fresh in her mind, so perhaps it was some deep-seated desire for revenge, or perhaps it was simply that proximity had given way to fondness. But Godric and Helga began to care for each other beyond the relationship of master and apprentice, and that fondness eventually grew into love.

She would watch, enchanted, as Godric and Salazar would argue about the origins of Magic, the blood of Atlantis. Long hours were spent debating whether, (as distasteful as such segregation would be to all of them), they should only allow entry to those full-blooded Witches and Wizards.

Although she did not take his side, she understood his concern. For magic to grow, it must be nurtured. Education was essential, on this they all agreed. Further, it was a well-observed fact that Magic begets Magic. Enough Wizards gather in one place, and the air becomes electric; ideas exchange more easily, Magic flows more freely. Enough wizards settles a land for enough time, and the land itself seems to change in response, with magical creatures and plants emerging, worming their way out of the collective subconscious.

They all knew and all agreed that the amount of Magic one carried was tied to their bloodline. But, they also knew and agreed that the amount of Magic one carries is not necessarily proportionate to the amount of Magic can output at any given time. Although Achilles may have a fraction of the endurance of the Tortoise, he can still sprint far faster over a short span. A school of pure-bloods would not ensure a school of powerful Magic, but it would ensure a high concentration of raw Magic.

So it came down to a simple question of quality versus quantity. Godric believed that it would be easier to find and teach 100 half-blood wizards than it would be find 50 pure-blood wizards. Salazar believed that this would present a logistical problem: how do you scale Hogwarts to handle that kind of population growth? Godric, ever the idealist, wanted to wait to solve that problem when and if it happened. Salazar, on the other hand, was known for the detail and care that went into his plans, and such an omission did not sit well to him.

The eventual compromise was to divide the school into three houses, with Salazar managing the pure-blood lines and the other founders managing the growth of the middle-bloods. They decided that the term “half-blood” was not strictly accurate, and carried pejorative connotations. Of course, they had not foreseen the linguistic corruption that would eventually shorten “Middle-Bloods” into “Mid-Bloods” and then twist that into “Mudbloods”. Nonetheless, they would compare results after a century or so, and that would dictate the course of Hogwart’s future.

Or, that was the plan, at least. This palace of education was a necessary evil, but its growth needed to be checked and stymied. Meldh still held the sting of Ollivander’s betrayal fresh in his mind. He knew he held no claim to her heart besides that which she freely gave. He still loved her, aged or no, and to see her with the newly-aged Godric Gryffindor caused him pain. Perhaps it was some deep-seated desire for revenge, or perhaps he was simply following the path laid down by Prophecy, but when the next steps were made clear, he gladly volunteered for the task.

Igbo-Ukwu
993 C.E.

Onyekachi was the leader of the Idemmli, and he had heard the legends. They were a people of conversation; their tales were passed from father to son, mother to daughter, from living mind to living mind, as it has always been. When sharing kola nuts, they would tell the tales of their past so they may live on for as long as the Idemmli lived on. But more importantly, they would grow, change, and adapt over time. The tale of Amiodoha had changed over the generations, and that was good. It diverged as well, for the women told a very different tale than the men. In one version, Chukwu formed Amadioha in his own image, and although they quarreled as father and son do, they eventually came together and defeated Ogbunabali. In yet another, Chukwu and Amadioha were destined to be rivals, and Amadioha rose up and overthrew Chukwu’s chains and went on to have a family of his own.

Onyekachi and his son, Ikenna walked hand in hand, both holding spears in their free hands that doubled as walking sticks. They did not have to walk long before they reached the place. He stared out at the small circle of Death in the plains, running his hand idly through the bare earth that stood in stark contrast to the lush grasses that were filled with creatures, plants, and tasty things to eat. It had grown slowly over his lifetime, and eventually it would take his entire village and people. But they, like Amadioha, would fight against its inexorable tyranny until they either won, or could fight no more.

But there would be time for that later. For now, it was time for his son to hear the tale, it was time for Ikenna to become a man.