{Pure Imagination}

Many years later
The Foothills Near Λείβηθρα

{A}{rchon} Heraclius Hero surveyed his army, his people. When he spoke, everyone heard, both mankind and God alike.

“The old Gods have ruled for too long. They have given us a hint, a taste of their power, and now we shall turn that power against them. For too long have they shackled the ambitions of mankind with their Magic. Man must rise of his own strength, and our strength is our mind. We, and those who come after us, must use this strength to build great things. We must not become bedridden, becoming reliant on an insidious crutch: this will-work that we do not understand, “gifted” to us by those who would use it to drive us into stagnation.

Magic, though it may be a blessing in some respects, makes us the play things of the Old Ones. If we rely on magic to build our empires, to grow our foundation, we build a house on sand. Our house must be built on rock, and that rock is our mind. A true curse is not one that brings nothing but misery. A true curse grants you power, a true curse is one whose pleasures are so intoxicating you do not wish to abandon it. Magic is that true curse.

Eperesto. Look at this. Sanguista. Is it not wondrous? Volesonorus. Is it not beautiful? That is the hallmark of its danger. Can you apply the principles of reason to these spells? Can you predict what words will cause what effect? If you cannot, you are a puppet. You are a plaything, stabbing in the dark, blindly grasping into places you know not with power that you know not. You are at the mercy of those who grant you this power.

Look around you. Do you see the glittering stone, the wondrous palaces, the plentiful houses, the water that courses through our city, the food that is bountiful upon your plate? That is the legacy of mankind. Look around you. What is the legacy of Atlantis? Do you live in a house that Atlantis built? Can you eat food that Atlantis has summoned? The old ones wish to keep us shackled, to keep us in their thrall, to damn us to millennia of darkness, subservient to them. I say “No!” We are subservient to no beings but ourselves.

Let us rise up! It is the start of a new dawn. No longer shall the world be ruled by muses and gods and fairies and Mystics. That is a world of stagnation, a world where we make no progress because no progress is necessary. You shall be the ushers of a glorious dawn, and history shall remember you brave souls as the true fathers of mankind!”

War requires planning. Careful, meticulous, well-thought out and well-executed planning. War also requires the ability to mercilessly discard those plans the second they were rendered obsolete by your opponent. Which of course, they always were. This has led many a glib commentators to suggest that the key to war is the ability to form new plans at a moment’s notice. Which in turn, has led many reactionary commentators to retort that the key to war is, in fact, the ability to create a master plan that is impervious to as many outside forces as possible.

The truth of the matter is that there is but a single winning move in the game of war, for both yourself and your opponent. As with Shatranj, being singularly focused on any one aspect of the game will ensure that you lose. You must consider everything in the context of that one, final, winning move.

The original plan was a variation on one of the classic formations. The core principle of magical warfare is that, assuming both sides execute perfectly, it is identical to non-magical warfare. A magic user is a force multiplier, and thus useless if you have no force to multiply. A single user can easily be overrun by a few hundred determined baselines.

That was one of the first hands-on lessons that Heraclius endured from his former masters, under the tutelage of the famed battle sage, Kobayashi. Hundreds of humans were given the protection of the Cross, and Heraclius was directed to defend himself by whatever means possible. Fire was worthless. It killed many, but it posed no physical barrier. Enough emerged through the wall of flame in fighting condition to force him to fight back with melee spells, and it was only a matter of time before a stray sword cut him down.

On his second attempt, he tried to construct physical walls. They were equally disastrous; the attackers simply poured over them like a river of angry ants. Nothing seemed to be effective. Widespread effects didn’t do enough damage to physically stop the onslaught. Focused, directed damage didn’t affect enough of the army to stymie the advance. It couldn’t be done.

The futility of the task couldn’t be the purpose of the lesson, otherwise they would have ended it hours ago. So he tried getting creative. At one point, he tried simply running away, but this also was not the answer that Master Kobayashi was looking for. His frustration was beginning to get the best of him as he tried increasingly outlandish gambits.

At one point, when they brought in a new regiment to serve as attackers, he decided to take a different approach. If he couldn’t overpower them with brute force, our outmaneuver them with sheer cunning, perhaps he could cow them with pure fear?

It was a new regiment; Master Kobayashi told them the instructions prior to the battle, but this was their first run-through. So Heraclius hastily assembled a crude simulacrum of Master Kobayashi, along with the Rosarius he carried. He hoped that the army was unfamiliar enough with the exercise that they would not realize that Master Kobayashi typically observed, invisibly, at a distance.

When the army began to charge, Heraclius began the show. He feigned an argument with his creation. He artificially magnified his voice so that the first line of soldiers could clearly hear him.

“I will not tolerate this indignity any further. I am one of the Descendants! I have the blood of the Gods flowing through my veins, and you subject me to this?”

The real Master Kobayashi would have said something wise, calm, and collected in response. And he certainly would not have cowered. but these soldiers did not know the real Master Kobayashi. All they knew was that they saw a tall, angry young Descendant towering over a frail, elderly teacher.

“Heraclius Hero, you shall not disrespect your masters with such talk. You will engage in the exercise.”

“I will do no such thing! Damn you, and damn your Cross! You take the gift of our Lady and you desecrate it by bestowing it upon these swine. You shall protect them no longer!”

He cast his hand out. Master Kobayashi’s Rosarius flew up in the air, in full view of the charging army. It imploded within itself, sending a shockwave out in all directions, knocking down the first several rows of the advance. Parlor tricks. Waddiwassi. Confringo. Ventus.

Continue the act.

He cast another hand up, and the frail simulacrum of Master Kobayashi was blasted forward, then engulfed into flames. He could hear the audible gasps of the men who were still standing. They were unsure of what to make of this. For a brief moment, it was silent except for the crackling fire.

Press the advantage.

He address the crowd, who had momentarily paused. “I will have no more of this. Flee now, in peace, and you shall live. Face me at your peril.”

The army shifted, uneasily. Master Kobayashi said they would be protected… But…

“Avada Kedavra!”

A bolt of green light, tinged with red and flecked with specks of violet shot through the air, striking one of the men on the front line. He fell, dead on the spot. His comrades were, in a way, relieved. The specific mechanics of the protection of the Cross were such that they would shortly know if their quarry was bluffing. They waited. Nothing happened. Their comrade did not move. The signs of the Cross did not come into play. Clearly, he was dead. Their protection was gone. And an insane wizard was now threatening to do the same to them if they did not flee.

So they fled.

As it were, their companion was not, in fact, dead, and as such did not invoke the protection of the Cross. The entire thing was a ruse: a falsified “Killing Curse” wrapped in a stunner, tinged with a Nexus Charm to mask the target’s vital signs. Another minute or so and the deception would have become apparent. But they did not want to wait another minute. And when the last of the army had left the battlefield, the true Master Kobayashi began clapping slowly.

Heraclius Hero breathed a sigh of relief.

The intended lesson was manifold. Firstly: a horde of sufficiently determined baselines could cut down even the most powerful of Descendants. Magic was not magic; it had its limits. But perhaps more importantly was that the will of the people could be easily broken. Even a crude and hasty deception instilled enough fear to turn them away. Fear was their greatest weapon.

“Teach them to fear us, and they shall never raise a hand against you. All that you do must be shrouded in mystery. Even your name must be something that fills them with dread. You need a True Name that inspires fear and raises questions.”

Without pause, Heraclius spoke. “Meldh. It shall be Meldh.”

Master Kobayashi considered this, and smiled. “An old word, yes. Many possible origins, and yet all of which point to the same undeniable meaning. Yes, that name shall do. You have learned your lesson well, Meldh.”

It was for this reason that, in magical warfare, the wizards were always stationed at the back. The front lines were too filled with randomness; a single stray arrow or sword could too easily turn the tide of the battle. From the rear, the wizards could manipulate the battle in relative safety.

Every attack had its disadvantages. Death from a distance was difficult to dole out, and easy to counter. Elemental forces had their fundamental opposites. Physical attacks could be turned aside. And clever gambits could be turned against their intended purpose easily. There were a few tried and true methods, but because their efficacy was so well-known, it was easy to plan against them.

It was Shatranj on a grand scale. You attacked, they countered. They counter-attacked. You countered. You enact gambits , you sacrifice material, you control your positions. And as with Shatranj, the game typically ends in a draw. Which means that the tide of battle is determined by the army, not the wizards commanding it.

The original plan was simple. The enemy was superior in size, so they would concentrate on breaking through a single weak point in the enemy line, in order to gain entrance to the Stronghold. Once achieved, the army parts, like the Red Sea, allowing the wizards and their specialized team of shock troops access to the Stronghold itself. Then the army closes back up, shifting into a defensive position. If successful, the terrain would not allow the enemy to bring the full force of their army to bear, and it would buy Meldh and his team sufficient time to complete their task.

The battle was brutal. The enemy wizards attempted to fill the sky with weapons of death. They were small, frail things, so Meldh summoned a fierce wind to blow them back towards the enemy. This counter had been expected, and the force of the wind cause the weapons to shatter. He expected those remnants had a secondary power unto themselves, and so he and his wizards cast fire into the sky, purging the air of the attack.

War wizards were well trained in turning their counters into attacks of their own, and Meldh’s team was no exception. Even before the fire had been cast into their air, one of the wizards was crafting wards the form of Amber and Lodestone. Once the fire had done its job, the wards contained and compressed the fire into a single, white hot point of nearly unimaginable heat, which was then directed downward towards the enemy army.

There were several fairly trivial counters to such an attack, but it was novel, a rarely seen combination of those three elements. As such, the enemies turned to Void, a relatively all-purpose means of containing an unknown threat. The air above the army crackled with the vacuum, as the fabric of the world itself was rent asunder. The Void was enhanced by the Boxes of one of their masters, and like the gaping maw of some creature that existed beyond space and time, it opened wide to swallow the singular point of energy.

Such a defense was not without costs, however. The Void swallowed more than the magical attack; it twisted the flow of the Ley, and all wizards both friend and foe alike needed to recalibrate. This meant that the defensive Void could not be harnessed into an attack of its own. The Void simply existed, was filled, and then existed no more.

Both sides opted for offensive, physical attacks once repositioned. Two volleys of enchanted arrows. Simple, effective, and ultimately nothing that a sizable team of archers could not accomplish on their own without the help of magic. The arrows were more effective against the defending army who had less freedom of movement. Attacker and defender alike were forced to raise their shields to ward off the projectiles, lending further advantage to the attackers who were not relying on heavy spears.

As the phalanx crashed forwards, the battle continued, a game of magical cat-and-mouse, for upwards of an hour. The enemy line was thinning against the continued onslaught, but it was beginning to compress inward once it had realized the nature of the attack. Time was now of the essence, so they began the second phase of their attack.

From the middle ranks, the men in enchanted plate mail charged forward. They held no weapons, only shields. This disoriented the individual defenders, who braced for attacks which did not come. This allowed the men, despite their limited mobility due to the armor, to slip by. They pushed farther into the ranks of the opposing army, using their shields as battering rams to continue the penetration.

The enemy quickly formed an interlocked defensive line of shields, which could not be penetrated by the handful of charging, armored soldiers, who were unceremoniously cut down. This was, however, the intended effect. The suits of armor, which had been linked together by magic, detected the moment that the last of their wearers expired.

And then they exploded.

It was not a particularly devastating attack, but it was unexpected, and it created a physical clearing which Meldh’s army immediately seized upon. Soldiers fled their individual skirmishes and flooded into the hole that was punched in the line, and pushed forward with reckless abandon. They could see light, grass, and rock. They had reached the back, the end of the line, and drove in like a wedge. The signal was called, and Meldh and his team began their charge.

What Meldh and his army did not know was that their attack would have been doomed to failure. It was filled with too many clever ideas and desperate gambits that had but a fractional chance of success. A well-placed explosion made for good drama, but a true army would quickly regroup and repel the attack with renewed vigor.

The attack was successful because The Three had decided it was time for their decisive strike against the coalition. The attack was successful because it was bolstered by the combined power of Gom’Jorbol of the Rod, and Kri’Xiang of the Glass. Masquerading as pawns, they led the charge into the Stronghold, and in the midst of the chaos, no one noticed the two lone soldiers charging inside before Meldh and the shock troops could arrive.

They were Gods, and yet, they looked like men. On the battlefield, they gave off no supernatural aura of power, nor did they make mere mortals feel compelled to bow before them through some unknown impulse. They were simply men. But once they were inside the Stronghold, it was time to put the masks on.

Gom’Jorbol stood, tall, proud, as his glittering armor instantiated around him. Massive, oversized epaulets crafted of plate were decorated with gilt and jewels. A stylized eagle adorned his chest, and a sash hung from his waist. He had seen the armor once, in a book from his youth an eternity ago, and it caught his eye. In his hand, he carried a two-meter tall spear, tipped with a sharpened diamond the size of two hands clasped together. The diamond alone was worth the riches of an entire kingdom. The combined wealth of a hundred kingdoms was not even a fraction as valuable as the staff which the diamond tipped: the Rod of Ankyras.

And yet, even that was but a paltry curio compared to the mirrored shield that Kri’Xiang carried in his hand. A massive, inviolate, golden oval that did not so much move through the world as the world moved around it. He was smaller than Gom’Jorbol, but no less intimidating when he chose to be. Together, they strode into the depths of the Stronghold.

Although they were prepared for battle, they found none. They were expected. Which was, unto itself, not entirely unexpected. When they reached the main atrium, they stopped, and Gom’Jorbol turned to Kri’Xiang.

“Do you think you can handle Janus and Kayla by yourself?”

Kri’Xiang laughed. “Do you think I’d be here if I couldn’t?”

“Good. Then let’s end this.”

They nodded towards each other, and went their separate ways.

As they walked down their respective halls, Kri’Xiang to the left and Gom’Jorbol to the right, they could hear the clashes of pitched battle begin anew at the entrance to the Stronghold: the shock troops had arrived. The distinctive roar of magic being used in melee combat echoed in the distance.

Gom’Jorbol reminded himself that if he survived this, he owed it to the mortals to help them escape. He pondered this thought, among many, as he walked down the hall. The familiarity of it all and the anticipation made it difficult to focus. What would he say? How could he do it?

He finally reached the tall wooden doors that led to the narthex. With one solid kick, they flew open, and he strode in, an avenging angel in battle armor, carrying the Spear of Destiny.

His heart swelled. At the end of the temple, stood Neirkalatia of the Cross, wearing armor of the same design with a distinctly feminine twist. She was beautiful, of course. He always found her beautiful. Physically, yes, but physical beauty was easy to come by. All of the Old Ones were stunning in their physical perfection, except the ones who deliberately chose otherwise. No, it was her mind and her Will that was beautiful. And the image he carried with him paled in comparison to the truth of her being. For the first time in eternity, Gom’Jorbol felt hope.

In her hands was a four-foot tall, stylized cross, a Masonic blade of supernatural sharpness and legendary unto its own right. Behind her was the True Cross, tall and glorious. Today, it was not wood. It was primed: the True Cross in its true form.

Today, Neirkalatia knew she was going to die.

She stood, her back facing Gom’Jorbol, staring up at the cross. She pondered the eons of time that had passed, and she thought about all the things she could say, all the things she could do. A small smile crossed her face.

“Hello, Gus.”

“Hello, Nat.”
