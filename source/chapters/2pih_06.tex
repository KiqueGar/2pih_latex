\chapter{Cups and Wands}

\lettrine{E}{very} scholar of magical theory knows that three is a magically powerful number. Now, there are certain disputes over why this is the case; some suggest that it has something to do with the physical pattern of the ley lines that connect the three major magical crossroads of the world. But the current fashionable theory of Functional Magic suggests that in a freeform, three\mbox{--}dimensional space where all else is equal, a triptych of nodes is the ideal configuration to most efficiently harness ambient magic. This theory has been backed up by several experiments and the principles of Arithmancy seem to bear out these results.
\SmallVSpace
However, anyone even remotely familiar with the tale of Harry James Potter Evans\mbox{--}Verres (who, depending on who you ask, is either the foreseen savior or destroyer of this world) would be well to doubt the veracity and rigor of these experiments. The fallacy of incomplete evidence immediately comes to mind. And indeed, the true scholars of Deep Magic know that the explanation is far simpler.
\pagebreak

As modern\mbox{--}day Slytherins know, three is simply the optimum number of people for a plot. One man alone is a crackpot, and would have much trouble converting others to his cause. Two is certainly sufficent; two can create the illusion of consensus and conspiracy, and can pressure a single person into action. However, only the most foolhardy of would\mbox{--}be plotters would devise a plan with no contingencies. If you are only Two, and something goes awry, you become One, and now you have no conspiracy to leverage. And because only a true fool would pursue a plot more complex than necessary, true plotters look for threes: no more, no less. As Saint Atilla, a master plotter unto himself, once said, “Three shall be the number thou shalt count, and the number of the counting shall be three.”

As such, there is always the leader, their trusted advisor, and a disposable confidant. As it was in modern times, so too was it in the ancient days. For as long as anyone with the capacity for memory can recall, there has always been The Three. In the beginning, it was Merlin of the Line, the leader, who was but himself. There was Gom’Jorbol of the Rod, the trusted advisor, who had appointed a mortal woman as his proxy and given her a measure of his Will, his Time, and his power. And there was KriXiang of the Glass, the disposable confidant, who went by many names, the most familiar of which was Topherius Chang.
\simpleline
It was in the ancient days that The Three began their plot. They began by removing the local leadership of Greece through a combination of spellcraft and outright assassination. Then, they stacked the local Thing with their pawns, and reached into the minds of the great philosophers and orators of the day.\\Finally, they took over the government by establishing the Eleusinian Mysteries. All things considered, a winning move was still a winning move.

They were opposed, of course, by a Coalition of Old Ones of less foresight and greater greed than themselves. The Three had a crucial advantage, in that they were willing to sacrifice themselves for their cause. And so it was that The Coalition had committed the third classic blunder. Any Guilderian scholar is well familiar with the first two blunders, but the third (significantly less well known) is this: “Never bring war against an opponent who has less to lose than yourself.”
\SomeVSpace
Despite this, in the first century BC, the Coalition performed a masterful coup, and their pawn Lucius Cornelius Sulla Felix deposed the Eleusinian Mysteries. A back\mbox{--}and\mbox{--}forth game of cat\mbox{--}and\mbox{--}mouse took place over the next century, with leader deposing leader, pawn fighting pawn, which ultimately ended it yet another seemingly decisive victory for the Coalition. But they placed far too much trust in their mortal pawns, and became far too reliant on their artifacts of power, which were anchored to this world and thus destructible.
\SmallVSpace
There was one pawn of the Coalition, who saw the glory of humanity, and envisioned a future where they were not enslaved by the whims of ancient manipulators. And in time, that pawn moved strategically across the board and was elevated by his masters, and became the regent of Neirkalatia of the Cross. He betrayed his master, took her secrets for himself, and in the name of Mankind, led his army against the Titans of the Coalition at the foot of Mount Olympus.
\SomeVSpace
Neirkalatia of the Cross, had waged a desperate and fearsome defense in the heart of her stronghold. In her desperation, she established a direct connection with the final Spire of Shiggoth, which in turn had a direct connection with the Central ley line. The power would, of course, eventually destroy her bodily form, but she would have sufficient time to end her attackers and ensure that her crux was properly bound.

But one does not tap into the anchor of Merlin of the Line without cost. Had she been more prudent, she may have gone unnoticed, and may have succeeded. But she was reckless. She poured all of her Will into establishing the connection, and as such, he became aware of the encroachment. He knew the time was right to sacrifice the Central ley, and in the instant he made the decision, all of its power was directed through the connection to Neirkalatia and every aspect of her, her Will, her Time, her Self, and her crux were burned through to the core.
\SmallVSpace
The Coalition fell that day.
\SmallVSpace
It came at a great cost to The Three. KriXiang of the Glass had sacrificed himself, after a fashion. His anchor of power, an incomplete and yet perfect reflection of itself, was turned upon two of the Coaltion: Yanotuk of the Cups and Kari of the Cube. KriXiang had sealed the three of them in a place beyond Time. The Three became Two, and the knowledge of a number of objects of terrible power were lost beyond Time as well.

It would soon come to be known that two aspects of Kari and Yanotuk had survived the Sealing. The Cup of Dawn, and a single Box of Orden. The loss of the Boxes of Orden was a blessing; the three of them combined represented such a vast destructive potential that Merlin had at times considered directly challenging Kari for control of them. The loss of the Cup of Midnight was a horrific tragedy; it was instrumental in one of his more crucial plots, and the lost centuries would ultimately account for billions upon billions of deaths. Yet another sacrifice.

But, Merlin also had Ελαολογος, the master artificer. She had arrived to Albion centuries before, after having successfully reproduced the Rod of Ànkyras. The original was as large as a stave, with multiple cores of several creatures whose properties lay in synergy with each other, and could easily amplify the caster’s power. When miniaturized, however, it’s power was greatly reduced. When reduced to a single core, it became, at best, a useful little tool for small bits of hedge magick. At worst, however, it was a crutch, and could potentially limit the magical development of an entire region.

Deep Magic is difficult. It requires the proper state of mind, the ability to hold multiple realities in one’s thoughts, to manipulate both in synchrony with each other. When cast properly, it can yield awesome, yet dangerous results. Many people have the potential for Deep Magic. Fewer people have the resources to pursue and cultivate this talent. Even fewer have the required skill to do anything useful with this, without years of training.

When using a Rod of Ànkyras, even a fledgling wizard can violate the most fundamental laws of nature and produce water out of the aether. Why would anyone bother to pursue Deep Magic, when such miracles were within the grasp of mere children?

Yet, when wholly reliant upon a Rod of Ànkyras, even the most powerful of potential mages will likely do nothing greater than summon living flame, or temporarily change the Substance of a Form for a matter of minutes. It was with this in mind that Ελαολογος, many years before, had left her lover and traveled to an unfamiliar continent and took a new name and made a new home, and eventually, started a new life.
\simpleline
\DatePlace{The Aftermath\\
The Foothills Near Λεíβηθρα}
It took over thirty\mbox{--}six hours, but they succeeded. He lost slightly over one half of his men, but they succeeded. He took an arrow to the shoulder, and suffered an inch\mbox{--}deep slice across his leg, but they succeeded. They had broken the lines of the Titans, stormed through the mountain stronghold, and destroyed the Third Tower.
\SmallVSpace
As a result, the Central ley line was lost. Creatures across the land blinked out of existence, those who relied on the ambient magic generated by the connection. More powerful creatures with their own nodes remained, but were diminished. The Muses and the Titans and the Fates and Furies narrowly escaped into another world.

The impact was felt as far as Egypt, where the priests of Ra and Anubis felt the power of their relics die in their hands. It was felt as far as the Arabian Peninsula, where Djinni died in their lamps. It was felt as far as Alto Alentejo, where the Falxian Priests could no longer feel the magic within their rock warrens. But they were free. Man was free to grow and develop a civilization.
\SomeVSpace
Albion, however, was still imprisoned. It had the Eastern ley and the Northern ley, that lay in crux with each other, amplifying their power to the extent that no Tower was needed to anchor it to this world. The peoples in Albion would be held in thrall for generations.
\pagebreak

Meldh strode through the camp, still feeling the glorious high of victory. He looked out among his people. He looked out among mankind. He smiled, because he knew that a new dawn was rising, a new dawn where a man would be free to exercise the fullest fruits of his mind; his capacity to reason. He looked out and he smiled for these were his people. He went by many names, one of them meant “protector of mankind”. Although he had long since discarded the name, he took the appellation seriously. These were his people and he was their protector, and they protected him.
\SmallVSpace
He dwelled briefly on the hypocrisy of fighting magic with magic. He dwelled briefly on the pain of loving his people but not trusting them. He quickly moved on, for trust is a deeper bond then love. A parent loves their child, as Heraclius loves his people. But, a parent cannot fully trust the judgment of their children; a parent will afford themselves certain privileges, certain rights that they cannot afford their children. So too was Heraclius the shepherd of his people. At one point in the past, he was one of the chosen, picked (perhaps capriciously) by the Old Ones to help them shape their vision of the world dominated not by man but mages. He was gifted with great power and lore. But he did not turn that gift against men. He was the Protector of Mankind, and he took that honor seriously
\SmallVSpace
As he strode through the camp, he looked upon his men, men who fought valiantly while many of their companions perished. It was, no doubt, a sacrifice, but importantly, they chose the sacrifice. He was not a ruler who would choose for his men. It was not his place to choose whether they should give their lives or not. He offered them the choice and they accepted, because they were men of honor, they were men of foresight, they were men of bravery.

One man had fought with such ferocity that even in the heat of battle, it had caught Meldh’s attention. That man had now discarded his battle armor, and was standing in front of a small fire, gazing into its depths, alone. He was middle aged, with a body that was at one point in peak physical condition but now wore the hallmarks of age like a badge of honor. His face was deeply lined. It was a face that had seen much. Perhaps too much. His green eyes were warm, though. Meldh spoke: “We have won a good battle here friend. You fought well.”

The man placed his hand on Meldh’s shoulder and replied. “But there are still more to be fought. You are a worthy leader. But I fear you may not yet be strong enough for the battles to come. \emph{Egeustimentis.}”

 

