\chapter{Huis Clos}
People expect old men to die,
They do not really mourn old men.
Old men are different. People look
At them with eyes that wonder when…
People watch with unshocked eyes;
But the old men know when an old man dies

Ogden Nash
\simpleline

Meldh stared into the abyss of the Lens of Kasreyn. It whispered stories to him, tales of other places, other times, other lands, other worlds. It told him tales of Horcruxes and Hallows, of love and betrayal, of nonsense and irrationality. It also told him tales of science and art, of love and friendship, and the failings of reason. He spent more and more of his days engrossed in these worlds.

Simply put, he was lonely. It was one of the occupational hazards of being functionally immortal. He wondered if this was how the Old Ones had felt, in the days when they were the hidden hand behind the machinations of the world. He wondered if this was why they amused themselves with silly politics and mean games, because they felt as he did: the only adult in a world of children.

And as they died, one by one, he wondered if they too felt the same sense of loss, of one more potential companion gone forever? By his account, there were only a handful remaining. There was, of course, Merlin, the once and future king. But there were others. One lay in eternal sleep, its endless nightmares giving weight and power to the Firelands, the realm of the Unseelie. Another lay imprisoned beyond an unbreachable seal. And yet another still walked this world.

Merlin, of course, had plans for them. He always had plans. They were as incomprehensible to Meldh as Meldh’s own plans were to a Muggle infant. The difficult part of most of their plans was the waiting, the interminable waiting. In the early days of the world, their was much work to be done, and it seemed that Meldh was always busy. Now, they simply watched as the clockwork machine of the world that they had wound so meticulously in the past centuries ceaselessly ticked away.

Their relationship had grown colder since the Battle of Hogwarts which resulted in Meldh’s death and subsequent exile. A century trapped alone with only his thoughts gave him ample time to consider the events of the last thousand years, free from any distraction or outside influence. He came to realize the intention behind it all, how cleverly he had been manipulated. ‘Cleverly’, because, even now that he understood, he still would have done no differently.

Merlin had warned him that they would do terrible things on the path to righteousness, that to save the world, they must destroy it first. But one of Merlin’s crucial flaws was that even though he made you do the right thing, and even though he made you recognize and acknowledge it as the right thing, you still hated him for it.

In Meldh’s youth, he had been taught by many wise philosophers, several of whom had proposed some variation of a dilemma that Meldh termed, “The Chariot Problem”. Consider a chariot racing out of control towards a crowd of people, and the only way to slow its advance is to throw an innocent person into its path. Is it right to sacrifice one innocent in order to save the many?

Meldh had rejected problems of this sort, as the world did not truly work like this. You were not presented with binary options of such black-and-white, clear-cut consequences. There were always unknown factors, always alternative options, and if you were brave enough, intelligent enough, cunning enough, or worked hard enough, you could always find a way.

Merlin had shown him his folly. He stripped away the illusion of complexity. He distilled the world down to its barest, most granular components. He illustrated with cold, brutal efficiency that sometimes you were, in fact, presented with a choice between the lesser of two evils, where the only alternative to that choice is ignorance: to evade the responsibility of making a decision.

“Such is the curse of competence. You understand, with full knowledge, the true extent of the consequences of your actions. Is it any wonder that so many prefer to consign themselves to blissful ignorance? And do you see what a monstrous crime that really is, if you are capable?”

Despite this, when faced with an unpalatable sacrifice, Meldh often tried to devise clever solutions. Merlin was merciless in forcing Meldh to fully confront the reality of the problem and evaluate his proposed solutions. A daring plot seemed much less noble when, upon the balance of probabilities, lives would be lost.

“Normal people do not live as you and I. They have but one life; their actions, their time, their resources, they are all limited. In order to win you must learn to lose, and this is a luxury they cannot afford. No, they do not play to win. They must play not to lose. It is not that they are irrational or evil, it is simply a matter of necessity. They play for different stakes.”

Meldh had learned first-hand how the entire fabric of one’s morality could be fundamentally altered in an instant when the stakes shifted. Before meeting Merlin, he would have done anything to stop the inevitable destruction of Magic. Now, bringing about that end was his life’s work.

He also used this to his advantage during his encounter with the three Peverell brothers, named in prophecy. They were desperate from their lack of progress in creating their weapons against Death. So they followed the whispers and the rumors, determined to defeat Death by confronting him on his own terms. They traveled to the Keep of Mysteries, unraveled the secrets of the Arch, and entered the Land of the Dead.

They stood at the foothills of a vast black mountain range and followed the shores of the gruesome lake that served as the headwaters of the Lethe River. Although the river was shallow, it was wide, and its waters flowed quickly. Many men had lost themselves to the river’s waters over the ages. The three brothers had studied the lore; they knew that to cross properly, they needed to construct a bridge of bone.

When they reached the other side, they saw it, a black figure composed of fractal shadows, folding inward upon themselves, and then blossoming outward in self-contained patterns. Despite having no constant form, no defined starting point or ending point, something about its essence seemed anthropomorphic and vaguely human.

Meldh watched them as they approached. Although The Land transcended physicality, one could still walk in if one knew the right path. The Chariots of Fire certainly provided some advantages; namely, it allowed one access to Tír inna n-Óc from anywhere on the planet. But walking into the Land of the Dead as a mortal had advantages unto itself if one could survive the inherently hostile nature of the place. When the three brothers walked close enough to Meldh to be within speaking distance, they stopped. Meldh introduced himself as Death and congratulated them on coming this far, offering them the gift of knowledge as payment.

The oldest brother desired a wand more powerful than any other, and he showed the work he had done with his crude stick crafted of Elder wood. Meldh revealed to Antioch the secrets of the Rod of Ankyras, showing how multiple cores could be made to lie in warp with each other, and demonstrated the precise structural manipulations needed to allow for consciousness to be imbued into the device. That living mind could pass knowledge surreptitiously from one owner to the next, but it also meant that it had intention, goals, and would not allow itself to be easily mastered.

The middle brother asked for the power to recall any mind from the eternal abyss of Death. The Spirit Stone was already capable of rebuilding a pattern from one’s memories, but the weaker the memory, the less accurate the pattern. So Meldh reached into a previously unused dimension and unfolded the True Cross, which was everywhere and nowhere. He taught Cadmus how to follow the fine traceries of the Ley Lines not just through Space, but through Time as well, in order to locate the essence of an identity amongst the oppressive noise, and reconstruct the pattern.

The third man, the youngest of the three, was also the cleverest. Ignotus had already created a True Cloak of Invisibility, his Hallow needed no improvement. He thought for quite some time, which may have been but a few seconds, it may have been several years. He had already concluded that their role was not to fight the final battle but to lay the groundwork. As such, he needed a way to ensure that the Hallows would find their way to the Crux when the time was right.

Meldh paused for a moment, the shadows within him writhing in time with his thoughts. They began to vibrate and warble, in a gesture that was unmistakably analogous to laughter. And at that, the shadows that comprised his form dispersed, and in their place, a white mist began to coalesce. Ignotus’ eyes widened as the form solidified into that of a man.
\simpleline

Cadmus’ eyes snapped open. He was in their bed, and it was still night. He didn’t want to disturb Ignotus, but the dream had been so vivid, and it disturbed him on a level that he could not quite describe. The principles made sense. He needed to test them. If you learned in a dream that two and two made four, it was no less valid than if you deduced it from first principles.

He quietly crawled out of bed, careful not to wake his husband, and slipped into their workshop. He removed his wedding ring from his finger, and tapped it with his wand in a slight corkscrew gesture, lifting away the Spirit Stone.

The next morning, when they met Antioch as they always did, his wand looked different. It felt different. It radiated an aura of judgment and immeasurable power. Without speaking to each other, they knew from a glance that the Deathly Hallows had been complete.

Although they never spoke of their shared dream, the legend of The Three Brothers still spread nonetheless.

\simpleline
\DatePlace{Alderney
1331, C.E.}
“Please, Master Payens, please. I’ve heard the rumors. I know that you know people, I know what people say about the Cross,” she gestured violently at the plain-looking wooden cross adorning the nave of the temple.

Cadmus was not listening to her. She was young, maybe 15 or 16 years old. She was speaking passionately about something or another. Judging by the small, frail body in her arms, her sister needed help. Or something. Cadmus was lost in thought, as he always was these days. He distantly observed that, had she been a little bit older, Antioch would have found her quite attractive.

He wondered dimly how she even found her way to this place. He no longer had the Cloak to keep him truly hidden, that must have been it. He found himself speaking a few words, and she responded, and he responded in turn. He had lost interest. He wanted her to go away.

“DONT MOCK ME!” , she screamed, the desperation apparent in her voice. How quaint.

“Oh? Or what?” He looked at her as she tried to form a response, then cut her off. “I know you, child. I have seen your personality before, in so many others. You see a problem in the world, and you burn with righteous rage. You hate the world for not fixing the problem, and you take the responsibility upon yourself, which you think justifies your impudence and rashness. Mark my words, child: it’s easy enough to ask big questions and make big plans. But to follow them through? What have you done with your short life besides angrily make demands of someone greater than yourself?”

He was lost in thought again, this time recalling a few months prior, his yearly visit to the cemetery at Godric’s Hollow.

“Hello, my love.”

As Cadmus spoke the words, he kneeled at Ignotus’ grave, laying the bouquet of flowers down at the headstone. He caressed his wand, feeling the knobby globes that stood out against the smooth, elder wood of its shaft. He idly traced the symbol of the Deathly Hallows in the ground as he sat.

He held his wedding ring in his hand, inset with the jet black, angular stone that forever whispered to him. He considered turning it over thricely but knew that the heartbreak would be too much for him, even protected as he was underneath the Cloak.

“Not much has changed in my life since I spoke to you last. The last of Antioch’s male heirs have joined him now… And joined you, I suppose. Iolanthe and Celia both took husbands, as well. Iolanthe to the son of Linfred of Stinchcombe, you remember him, the potterer, and Celia to Greybold Gaunt. Iolanthe Potter and Celia Gaunt. There’s no one left to carry on our name. I am the last Peverell, and will be the last Peverell, for my heart is claimed, now and forever.”

His voice cracked as he spoke, and the crack widened into an open sob. He crumpled to the ground and wrapped his arms around the grave. “It should have been you, it always should have been you. I was a good man, but I never was a great one. I merely stood on the shoulders of giants. I was never strong enough to hoist the world on my back, or the pass the torch of knowledge to all of man.”

He sniffled, regained his composure, and spoke again. “I’ve thought about it a lot, our family crest. The last enemy… It’s as much of a warning as it is a challenge, isn’t it? Death must be the last enemy that is defeated. Until then, of what use is everlasting life?”

He slipped into his native tongue for a moment. “Le paradis, c’est les autres.”

“Other demons still stalk this world. Any student of the occult with a flexible moral center can stave off death for centuries, if not millennia. You don’t need the Elder Wand to defeat any foe. You don’t need the Spirit Stone to converse with the memories of the past. You don’t need the True Cloak of Invisibility to remain hidden.

“I came here to say… That it’s time, I suppose. It’s time for me to pass the Hallows on to someone more worthy. The Stone, I will gift to Celia, and the Cloak to Iolanthe. The Wand, of that I am still unsure. I fear that…”

He paused for a moment. He had visited Ignotus’ grave every year since his passing. At first, he felt a bit self-conscious over talking to an inanimate object. But he wasn’t really speaking to no one. No sane, rational being could ever look at the way magic works, observe the universe around them, and conclude that death was the true end of things. Maybe, perhaps for Muggles. There had never been a documented case of Muggle resurrection or Muggle immortality. But they had Magic.

That had been another subject over which Cadmus and Antioch had continually argued. To Antioch, the answer was self-evident: Muggles don’t have souls, and wizards do. It was why Antioch was so staunchly against the interbreeding of wizards and Muggles: as long as there was still a spark of magic, as long as their was a soul, Death was not the end.

Cadmus, on the other hand, took a far more reductionist view of identity. To him, the patterns that made up a person persisted throughout the echoes of time, Wizard or no. It was simply something about Magic that made those patterns more readily identifiable, easier to locate, easier to recreate.

This was not to say that Antioch was prejudiced. Quite the contrary: he believed they could not truly conquer Death until they could conquer it for all of mankind. Antioch spent his days with the Elder Wand trying to master the ultimate power, the ability to create life. True, soulful life. And one day, when the time was right, he would grant the blessing of a soul on ever non-magical man, woman, and child.

Their ends were the same, if not their means. Cadmus also sought to save everyone. But while Antioch lived for the future, Cadmus dwelled on the past: he endeavored to use the Spirit Stone to call for the lost souls of all, regardless of whether they were marked with the touch of Atlantis.

In retrospect, Ignotus was the wisest of them all. He sought to hide from death, to prolong his fate, realizing that they were not the chosen ones. They had been born at the wrong place, at the wrong time for what they sought to accomplish. The world was simply not ready. And so he remained hidden, in order to pass the Hallows on to someone who was truly worthy.

In the days after his death, Cadmus wept at the thought of his true love dying a failure. But he had not failed. Cadmus was now the sole and true owner of the Deathly Hallows, which meant he could pass them on as he saw fit. Celia and Iolanthe had proven themselves to be good people, to be worthy. There were a handful of remaining Peverells, in blood only, but they had not shown the necessary qualities.

But it still left the problem of the wand…

He thought back to that terrible day, comforting Antioch as he wailed in abject misery, his huge arms holding the mangled corpse of his equally huge friend, Osgurd. He had died at Antioch’s hand. It was another one of their tavern brawls and overcome with the song of battle and rage, Antioch drew forth the Elder Wand and was consumed.

Antioch begged his brother to kill him, for he knew that he could not control the Wand’s power. The Wand craved mastery, dominance. It hungered for an owner who could harness its power without being overwhelmed. And when Cadmus took the wand from Antioch’s hand in order to ease his burden for a moment, he understood.

The wand sang a hymn of battle, of struggle, of a profound joy resisting the indomitable shackles of death and suffering. It cried out in passion for an owner who could not just deal out death, for death was anathema to the true intent of the wand. It needed an owner who could right the inevitable wrongs that must be committed along the Path. It required a master who not hesitate to sacrifice one man to save ten, it also required a master who would not rest until that sacrifice was made right, made whole.

Antioch was not that man. The wand seemed to recognize this, and so it constantly tested him, put him in situations that would prove his unworthiness, allow his anger to take control. It screamed for freedom, freedom from the hands of a master who could not provide the balance it so desperately needed.

When Cadmus put his hands upon the wand, it joined with him and spoke to him of the Path of the Scorpion and the Archer and what they could accomplish together.With a single look into his Antioch’s eyes, Cadmus saw that his brother understood as well. Without a word, he slit Antioch’s throat, and so death took the first brother for his own.

Ignotus had always known that there existed far greater objects of power in the world than their own Deathly Hallows. They had heard the whispers of the Old Gods, the true survivors of Atlantis, and knew they must have had Hallows of their own to bind them to the world. With the power of the Elder Wand, the Spirit Stone, and the Cloak of Invisibility, Cadmus and Ignotus travelled the world together, growing their collection of lore.

From the holy land, they had rescued the True Cross and the Holy Grail. These of course, were superlative titles, as they bore little resemblance besides in appearance to the myths after which they were named. They navigated the ruins of Alexandria to find the Mirror of Noitilov and traversed across an ocean to the new world so that they would know the Gate and thereby claim one of its aspects. They traveled south to the ancient ruins of Fajin and defeated the army of Inferi in order to gain control of the last remaining Box of Orden.

As they gathered these artifacts, they consolidated them in the island of Britain and began to take various measures to protect them. When they had finally reached the end of the line, Ignotus, who was already growing frail, died unceremoniously in his sleep, his Cloak folded neatly at the foot of the bed.

After burying his love, Cadmus’ days of adventuring were over. He began to study the ebbs and flows of time; he gazed into the stars with the centaur flocks, he studied Cartomancy and Tasseography, and he unleashed the Words of Power and Madness in order to peruse the Web of Prophecy within the Keep of Mysteries. The more he studied, the more clear it became: these were the middling days of the world, and the end times were centuries away. The Muggles would devise a magic of their own, and those two worlds would narrow into one. And when push came to shove, the combined minds of billions upon billions was a magic and power far beyond the scope of anything he could ever hope to accomplish by himself–

–He was interrupted from his thoughts and pulled back to reality by the sound of a door slamming dramatically. He sighed, closed his eyes, and thought of the stars once more.
\simpleline
This was useless. Just another jaded power-hoarder. Damn him, damn his entire Order, damn his Knights, damn his Cross. God damn every last one of them. She would tear the world apart. She would rip apart the gates of Heaven, tear apart the very foundation of Christendom to pull her sister back.


