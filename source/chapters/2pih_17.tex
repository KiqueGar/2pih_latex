\chapter{Mad World}

Bored bored bored bored bored

Boring boring boring boring

Ordinary, normal, boring people little ants in the afterbirth nothing but ants and flyers little mouches, moochy too much too much too much just can’t

BORED

Koschei the Deathless strode restlessly across its chambers. It had lived a thousand lifetimes with a thousand different names and each one was the same: boring. It had tried being a hero. Boring. It had tried being a tyrant. Boring. It had tried being a man. Boring. It had tried being a woman. Boring. It had tried being a king, a queen, a prince, a pauper. Nothing. It felt nothing.

All things were within the grasp of Koschei the Deathless. It had met all the interesting people in the world. It had read all the good books, and then written books even better. It had celebrated its first grandchild’s tenth birthday party in the new world, it had celebrated its first great-great-great grandchild’s hundredth birthday party around the fairy rings of Stonehenge. Still nothing. Always nothing, always bored.

When all things were possible, nothing had meaning.
\simpleline
\DatePlace{The forests outside Череповец, Вологодская область
February 2, 1333, C.E.}

The stench of sex and blood was thick in the air, affronting the nostrils of the lone traveller. If he were with Muggles, he would be cutting through the wild gorse with his shashka, but it hung, unused, on his belt. If he were with wizards, a few well-placed Reductos would clear the path, but his wand was in its holster on his wrist.

This traveler was alone and had no appearances to keep up, for now, and as such, the path cleared its own way, saving him also the trouble of locating his quarry. The smell would have been enough, but easier is always better. As he drew closer, sounds began to mingle with the scents to form a two-pronged assault on the senses.

Moans. Shrieks. Wails.

Pounding. Thumping. Banging.

Flesh atop flesh. Bows across strings. Lips upon horns.

He approached the small cabin and glanced at the awkward stilts that held it above the ground. They were disguised with a small and silly glamour to look like the legs of a chicken. He paused for a moment, deciding how best to enter. Sometimes, dramatics were useful tools to achieving your ends. But sometimes, they backfired. What would the consequences be? And what were the consequences of his hesitation, however slight?

Every decision was like this. Every minute, every moment, was another moment in which his enemy was allowed to persist. Even the fractional amount of time it took to pause and consider the question, “To knock, or not to knock”, was another dread deed, another bit of senseless evil.

Every decision. It was torture. Time, time was of the essence. And so he entered.

The scene was ridiculous. Caligula would have been proud. Or more likely, he would have been envious to the point of rage. Every possible indulgence was being fulfilled. There was sex, of course. Always the obsession with sex. But if it gave them a moment of solace, why begrudge them? Every reasonable iteration of sexual combinations was currently being explored on almost every available surface within the grand hall whose interior was far larger than the simple cabin’s exterior.

There was food and drink, as well. Food and sex. Drink and sex. Food and drink and sex. Sex with food and drink. Drinking food. Drinking sex. Food and drink and sex and then more food and then more drink and then more sex. A swirling miasma of what should be “pleasure”, and yet, he was struck by the hollowness of it all. Did they truly enjoy this? Did Max truly enjoy this?

Then again, if they did enjoy it, what did it matter?

There were important people in this universe. People upon whose actions his plans were contingent. These people needed to be closely watched, guided, mentored, or in some cases, manipulated or coerced, into following the correct path. These people, these cruxes, were few and far between, and he was thankful for that, as he had spent much of the last few centuries guiding them through the eye of the needle. The universe did not permit more than a few kings and queens upon the chessboard.

There were also influential people in this universe. People are resources, put simply. Two people have more absolute potential than one person, but that potential is not always exercised. Those of influence, more often than not, pushed people towards one extreme or another. He saw them at every scale: globally, nationally, locally, socially. And those influential people themselves needed to be influenced, but that was easy enough to do behind the scenes with a hidden hand. A war here. A social movement there. Sprinkle in a few shifts in cultural direction.

Then, there were effective people in this universe. Not necessarily creative thinkers in their own right, but actors, capable of putting a plan into action. These were the pawns, the rank and file that were sacrificed without much thought. But, (as he always reminded his protege), a pawn could always be promoted to something greater, so they were not to be summarily dismissed.

Finally, there was everyone else. People who would live their lives and die without any measurable impact on the course of the universe. What purpose did they serve? He could spend half an eternity converting every single one of them, and it would do nothing. Little would be lost if they were gone. And yet, little would be lost if they remained. He was as a God, but he was not malevolent.

And if this was how some of them filled their small lives, and it brought them pleasure, why begrudge them that?

But there was one, an aberration, someone who, like him, didn’t fit the pattern. It sat at the head of the hall, on an elaborate throne, watching the proceedings languidly. She was beautiful. He? It? He could detect the Glamour, prismatic and ever-changing, attempting to probe his mind. Its intent was to determine what one found most deeply and profoundly attractive, and then subtly present that back to the viewer. But it was still magic, which meant it had limits.

Merlin of the Line was that limit, and he had reached his. “Max.”

The beautiful anomaly raised her head. “John.”

They could have called each other by a thousand different appellations or epithets. But there were no pretenses to keep up, no battles to fight. The battles had already been fought, and Merlin had won them all.

“It’s time, Max.”

At this point, all the Glamours had melted away. The beautiful people who were splayed about the floor in indulging in various ecstasies were dismayed to see their platonic figures melt back into the flabby, second-rate bodies of peasants and adventurers. They looked around, ashamed of their nakedness, and self-consciously began to skulk out of what was now a simple cabin in the woods.

The two Ancients ignored them. “I knew you’d come for me eventually. I’ve been expecting it ever since the Interdict. Which, I have to say, I don’t quite understand.”

Merlin cocked an eyebrow. “Oh?”

“I never played the game on as many levels as you. I never had the need to, and I never had the want to, either. The games bore me, and if we’re being honest, the world will move on without me. I know you. You’re going to shape the world how you want it to be shaped and there’s not a god damned thing any of us can do about it.” At this, Max idly spat on the ground. “It’s why we’re all here, and not there,” he added, bitterly.

A beat of silence passed. None of what Max said required a rebuttal or response, so Merlin provided none. Besides, it was clear Max was mostly thinking out loud, and it was not long before he continued. “The first level interpretation is that you saw the danger of magic and did something about it. Only a fool would accept that at face value, which is why the majority of the world doesn’t look farther.

“The second level is that it’s part of a larger plot, the first move in an epic, century-long war of attrition to eradicate magic. Of course, the hypocrisy of that is blatant: using magic to eradicate magic? That’s something that a villain out of storybook would do. And that’s where I’m stuck. You’re not a storybook villain. And tactically, it doesn’t make sense. If you have that kind of power at your disposal, and magic is your enemy, why limit it in this oddly specific, easily circumvented kind of way? There’s another level here.”

Merlin began to smile. It was a slow, sad smile, but it carried with it a hint of amusement. “I thought you said that the games bored you?”

“So it is part of the game, then.”

“Isn’t everything part of the game?”

“Depends on your definition. The game itself bores me. But the meta-game does not. Like I said, I’ve been waiting for this for centuries, to see what you have planned for me. It’s really the only thing that I’ve looked forward to, the only thing that has kept me going.”

“Then what I have planned for you will be poetic.”

Another beat. Max spoke, “You want me to die.”

“We all must die, in order for the world to live.”

“You know as well as I do that there’s no middle ground, here. Either everyone dies, always, and forever. Or everyone lives. Always and forever. Infinity or zero. Nothing in between.”

At this, Merlin smiled. This truly was the crux of everything.

“You said you’re bored? Well, there’s your riddle. Figure out what I want, and then do it. Because it’s going to happen, one way or another,” Merlin paused, briefly, and then turned to leave. As he opened the door and stepped out onto the stilted porch, he looked over his shoulder. “It’s good to see you, Max.”

“You too, John.”

And for the first time in millennia, Max Koschey, Koschei the Deathless, Baba Yaga, Ma’krt of the Rock, He-With-A-Thousand-Names and a thousand other names, was interested in something.
\simpleline
\DatePlace{Hogwart’s Castle
June 13, 1334, C.E.}

“You BITCH!”

Her world was ice. Her world was crystal. Her world was fire, burning through every metaphor until nothing existed of her but the abyssal depths of her dark side.

“Crucio!”

She felt nothing.

“CRUCIO! CRUCIO! CRUCIO!”

Her breath came in ragged pulls and she poured all of her magic into the pain. Still, nothing.

“YOU FUCKING BITCH!”

She reached for the nearest heavy object, a candlestick on the nightstand. She was still naked. They both were. Normally when she was exerting herself, her hair would come loose, cover her face, obscure her vision. But today, it was slick with sweat and blood, and stuck to her back and chest.

She swung the candlestick, hard.

“This is for my mother!”

She swung again.

“This is for my father!”

CRACK.

“THIS is for Babette!”

The candlestick finally snapped. At this point, what she was swinging at was an unrecognizable, pulpy mess.

“YOU KNEW. This entire time, you KNEW! This entire time you could have done SOMETHING. ANYTHING!”

She choked out a sob. With no convenient weapon, and almost no magic left in her, she resorted to her fists.

“God damn you. GOD DAMN YOU.”

Impossibly, the breaths still came. She knew there was one last thing to be done, and she had held a small part of her magic in reserve. She hoped it was enough. With an angry cry of effort, she plunged her fist, augmented by a small flow of magic, into the chest of her victim. With a wet sucking sound, she pulled out what she sought.

A green, fist-sized chunk of crystal. The Heart of Koschei the Deathless.

She had a speech written in her mind, about the millions of deaths that Koschei was responsible for, and the blood on its hands and the good that it could have done and the choice of inaction and the path of evil and her own grand dreams and ambitions and how she would change and save the world. But she could not form coherent words, only vitriol.

“You… fucking.. BITCH.”

She held up the Heart. It was poetic in a way. She would use its own power to destroy both the Heart and its owner. It would, of course, be diminished. It would be a sacrifice. But it would be more than sufficient for what she hoped to accomplish.

She used the final mote of magic left in her to transfigure the Heart into something lesser. It was smaller, the size of an egg, and it was no longer the brilliant, iridescent green that reflected an infinite multitude of colors while still maintaining its own identity. Now it only reflected what was on her mind: dark, ruddy, sticky blood. She tapped into the power of the Heart.

Its form was Changed. As too, was the God beneath her. An instant before, it was a broken, but living, breathing person. An instant later, it was a corpse. It was over.

And that was the tale of Koschei the Deathless.
\simpleline

\DatePlace{Hogwarts Castle
Nine Months Earlier}
“Nell!” She pretended not to hear him.

“Nell!” Nope.

“NELL!” She kept her head buried in the book.

“Don’t make me send a Howler over there!” She rolled her eyes, and briefly glanced up over the top of her book. “Whatever.” That red-haired git of a Weasley, somehow had grown handsome in a silly sort of way in his sixth year. He was still tall and gangly though. And he had a stupid name. Festivus. “Can I help you with something?”

Festivus’ companion, who up until that point had been eying Nell’s friend sitting next to her, chimed in, “Oh, I think he needs a lot of help.”

“That’s certainly true, my dear, but I come with the noblest of intentions. See, I read in a book once–”

She cut him off. “YOU? You read a book??”

“Don’t get too excited. Bewitching Witches and Ways To Woo Them. Brilliant, if I do say so myself. It says that the only thing women want to do is to talk about themselves and that the greatest gift you can give them is your ear.”

His friend wise-cracked once more, “I don’t think there’s a big enough box to fit those things. Unless you plan on dropping her off of the side of the Tower and letting her use them as parachutes!”

“Shut up, Ollie. Can’t you see that I’m winning her over with my charms? If you–” Nell interrupted him. “Oh, I’ve seen you cast charms. And I think I’d rather hear that Howler than watch that again. If you must know, I’m currently researching the edge cases surrounding exceptions to Gamp vis-a-vis the substance-form dichotomy, specifically concerning the influence of mind altering spells such as the Confundus Charm and Geas.'”

Nothing. Just a blank stare. She rolled her eyes. Gryffindors.

Festivus blinked a few times. Ravenclaws.

“Cool! Well. I just got done putting a little something special in the pumpkin juice. So forgive me if I’m not impressed by your less lofty pursuits.”

“Go away before I Geas YOU. I’ll make you think Ollie here is prettier than I am!”

Ollie couldn’t resist the obvious joke. “You know, I’d like it if you made Helena think the same thing!” Helena blushed furiously. Nell feigned a look of confusion. Festivus gave Ollie a sharp jab in the ribs with his elbow.

Git, Nell thought.

Git, Festivus thought.

Ollie was busy thinking about Helena.

Helena was busy thinking about–

–“Watch it, here comes Headmaster Gag-Me,” Festivus whispered under his breath, breaking the awkward silence.

“Good morning Festivus, good morning Grumblechook! I trust you had a productive summer!” Headmaster Gagwilde strode in, interrupting the conversation with his usual dramatic flourish.

Grumblechook Ollivander rolled his eyes: he hated his name. His mother said it was an old family name, but he secretly suspected that she lost a bet with her brother-in-law. “Ollie” was just fine as a nickname. While Festivus and Ollie had a perfunctory conversation with the Headmaster, Nell briefly pondered wizarding genealogy.

It was long rumored that Godric Gryffindor had an illegitimate child with Galath Ollivander hundreds of years earlier. That child perpetuated the Ollivander name and bloodline by having male child after male child after male child. That is, until Genevieve, the only daughter of a mother who died in childbirth.

The Ollivander bloodline had to be preserved, for obvious reasons. But so too did the Ollivander name; it was good for business, after all. As it so happened, a distant cousin of the Ollivander line had given birth to a baby boy: Garrett Goyle. His mother too had died in childbirth, and the had abandoned her months before that. So it was that the Ollivander family adopted Garrett Goyle, who became Garrett Ollivander. He eventually married Brunhilda Nott, and the Ollivander name endured. Genevieve Ollivander married Septimus Weasley, and the Ollivander bloodline endured.

False-brother and false-sister had their respective children on the same day: Festivus Weasley and Grumblechook Ollivander, and the two hadbeen virtually inseparable ever since. By blood, they were not even cousins. But despite this, people called them “the twins”. They did everything together. They were so close that they often finished each other’s–

“–sandwiches?”

Nell’s concentration broke, and she looked up. Festivus had scooped up a particularly disgusting looking plate of sandwiches and offered one to Nell and her companion. She grinned. “No, thank you. Really. Did the house elves make that sandwich? Or did you make it out of house elves?”

“Who can tell, anyway, with last year’s crop? Well, I’m off to go stuff my face. Enjoy!” And with that, Festivus departed. As he walked away, he turned back over his shoulder and called back to her, “Oh by the way, steer clear of the pumpkin juice!”

Helena was still blushing. “You know, I don’t… I don’t think you’re pretty. I mean. No. I don’t mean you’re not pretty. I mean. Oh. I, uh…” She blushed even harder and looked down at the table, stammering.

“Helena. Helena. It’s okay. Really.” Nell put her hand on Helena’s. “Really.”

Her hand stayed there. For a brief moment, she looked directly into Helena’s eyes and smiled the smallest of special smiles.

Perenelle du Marais’ parents were healers. Making people feel better was in her blood, and it came to her naturally. “This world is a broken place,” her father reminded her, constantly. “It is our role to fix it.” Every day, she reminded herself of her goal and strived to wear the mantle her father had passed down to her.

Because they were healers, the accident was all the more tragic. Perenelle had a sister, once. A sister who, like her, was so full of light, and wanted nothing more than to be just like her father, and fix the world. A sister upon whom she doted, and who adored her. Wizards are preternaturally resilient, but even mundane things can take their lives, if help is far enough away, or the condition is serious enough.

Sadly, modern techniques such as cardiopulmonary resuscitation were unknown to wizards in the 15th century. Lungs filled with water were notoriously difficult to treat. Her parents tried desperately to coax the liquid from her but to no avail. In her desperation, Perenelle transfigured the water into a different Substance. She knew that if the transfiguration broke, it would be instantly fatal. Perenelle was only a few years in to her education and struggled mightily to maintain her Magic. Her parents knew better than to hold out false hope, even though she screamed at them in rage, imploring them to help her, even as her Will faltered. As she held her sister in her arms, she poured everything she had into it, and more.

It was not enough.

Her parents passed in her fifth year, victims of Dragon Pox. She would later learn that a cure had existed for centuries. The world was saturated with stupid, senseless deaths. The world was broken, and she intended to fix it. Even if she had to break it first. Over the years she had heard whispers, old tales of artifacts and Gods from a bygone era, stories of lore beyond reckoning. In the summer of her fifth year, she left her native Alderney and traveled the old world. She visited the marble edifices of Alto Alentejo. She saw the tombs of Egypt. She spoke with the wraiths of Białowieża. She was still young, so young, and thus collected no more than whispers, murmurs. But there was one murmur that rose louder than the others.

The mass of students in the Great Hall murmured. Another Dark Lady to teach Battle Magic? But Morganna was one of the best professors that Hogwarts had seen in generations!

Headmaster Gagwilde stood at the podium at the forefront of the Great Hall, delivering his beginning-of-the-year address with the affected, eccentric pompousness the students had grown to know and love. “Yes, it is true. Our beloved Professor LeFay has departed Hogwarts, leaving us with a vacancy. Fortunately, Professor Ollerton was doing a bit of adventuring in Poland over the summer and convinced a new Dark Lady to share her lore with us. Witches and gentlewizards, allow me to introduce you to our newest Battle Magic professor, Miss Baba Yaga!”

Any student who had been drinking pumpkin juice immediately spewed it from their mouths in a fantastic synchronized spit-take, prompting waves of laughter to ripple through the Great Hall. Baba Yaga? Headmaster Gagwilde was famous for his jokes. This had to be one of them.

Festivus Weasley and Grumblechook Ollivander, for their part, were particularly proud of their ingenuity. Comed-Tea in the pumpkin juice? Classic! Helena Ravenclaw, who had been smiling almost uncontrollably to herself prior to this, looked over at Perenelle. Normally, she too would be grinning, despite herself, at another one of Festivus’ stupid pranks. But instead, she had the Look. That look that Helena had come to recognize from their years together at Hogwarts. Long years, spent watching. It was the same look Nell had when you asked her about her parents. Or her sister. Her Dark look.

“Nell? Are you…” Helena considered putting a hand on her shoulder, but thought better of it. Nell blinked a few times, and the smile returned to her face.

“I’m fine.”

