\chapter{Ms. Phaethon}

“He lived to see the night which, by the accepted laws of history, he was not supposed to see. He was forty-three years old and it was the opening night of Phaethon, an opera he had written at the age of twenty-four. He had changed the ancient Greek myth to his own purpose and meaning: Phaethon, the young son of Helios, who stole his father’s chariot and, in ambitious audacity, attempted to drive the sun across the sky, did not perish, as he perished in the myth; in Halley’s opera, Phaethon succeeded. The opera had been performed then, nineteen years ago, and had closed after one performance, to the sound of booing and catcalls. That night, Richard Halley had walked the streets of the city till dawn, trying to find an answer to a question, which he did not find.”

Atlas Shrugged
Ayn Rand
\simpleline
\DatePlace{Hogwarts\\
June 11, 1334, C.E.}
\lettrine{I}{t} was very strange, seeing Ollie actually being sincere. When he was paired up with Festivus, (which was to say, always), it was an endless stream of jokes, wisecracks, and laughter. But Ollie was nowhere near as sharp or subtle as Festivus. He just had a hard time gauging people’s reactions to things. So he usually went for the obvious jokes, which actually worked out because he served as a good foil to Festivus: Ollie was Il Capitano, and Festivus was Arlecchino.

This was a shame because Ollie wasn’t actually dumb. In fact, as Nell had come to find out, he was actually quite brilliant. It’s just that he wasn’t particular good with boundaries. He never quite knew where the line was and when or if he crossed it. Nell definitely understood that, albeit in a different sense. Nell was pushing the boundaries of the world itself, and she wasn’t quite sure when the world would finally start pushing back.

As of yet, it had not. The path seemed to simply unfold for her. In this case, it was simple enough, by her standards. Since the start of term, Ollie had been hinting, not-so-subtly, that he wanted to ask her a question. Alone. She and Helena were nearly inseparable, as were Ollie and Festivus, so a private moment between the two of them would not go unnoticed by the rumor mill of Hogwarts. This was something Ollie definitely wanted to avoid.

And so, Nell did as she often did; she killed several birds with one stone. She needed to take advantage of his familial connections, and procure a particularly valuable artifact for a couple of days. And he needed to have a plausible excuse for speaking to her alone. Of course, Nell always had several irons in the fire, so she needed quite a great deal more than just a simple rock.

She was a spider, and her web was manifold. Nell and Ollie were meeting up after class on Tuesday. But why? For the majority of Hogwarts, they either didn’t care, knew it was not their place, or it was simply none of their business. But for anyone who asked, there was the first-level lie: they were building some sort of Super-Cauldron to give themselves an advantage in Professor Rothtim’s upcoming Potions exam.

The story deliberately had a tiny hint of inconsistency, just enough to cause an astute meddler to dig deeper: Professor Rothtim gave four exams per semester, but only counted the highest three grades from among them, so as to not unduly punish someone for a single poor performance. This was the fourth exam, and Nell had aced the previous three.

Why would she even bother taking the fourth exam? Well, she was Nell, after all. Taking extra tests just to take them was totally a Ravenclaw thing to do, right? But, then, why would she be going to all that extra effort? She aced the previous tests without needing a super-cauldron… This doesn’t make sense. I better ask around.

Nell was leaving a trail of breadcrumbs. The story that she whispered to friends and select confidantes (which meant, of course, that it might be kept a secret for a handful of days), was that she had discovered a potential loophole in Bertrand Whitehead’s Principia Discordia, specifically concerning the theories of Magical Recursion, and she needed to test a practical example.

This explanation reached the ears of the envious and meddlesome, some of whom practically fell over themselves in order to try to procure a copy of her notes. Nell happily obliged them. She had long before claimed a corner of the Ravenclaw common room as her own little study nook, and routinely left her notes and personal effects scattered about the desk. No Ravenclaw was brave or foolhardy enough to even dare to steal from her.

And indeed, no Ravenclaw would. Elsa Greengrass, a Slytherin who was also in her sixth year, did not particularly care for Nell. Elsa was pretty. Some might say she was prettier than Nell, but Elsa thought of herself as second-rate in comparison to Nell’s effervescent charm and her slender curves. Elsa was also quite intelligent, but she thought of herself as a dolt when she looked at the countless breakthroughs Nell had made throughout their years together.

Nell had, on several occasions, tried to befriend her. But Elsa was a Slytherin, through and through. She was born and raised, steeped in a culture of cunning and intrigue. Everything was a plot, everyone had ulterior motives, and nothing could be trusted at face value. So she always scorned these offers of friendship, because how could she ever trust another person who was capable of being her equal? Or worse, her superior?

The irony was that Nell was actually more of a Slytherin than she ever let on. Everything she did was a plot. Everything she did had ulterior motives. Nothing she did was purely face-value. And yet, she was not a Slytherin. At the end of the day, when all of the plots had reached their resolution, with no rationalization necessary or self-deception, she honestly, truly was trying to fix this broken, broken world.

But Elsa didn’t see that. All she saw was the Slytherin-in-Ravenclaw’s-Colors who managed to have the whole of Hogwarts wrapped around her little finger and did whatever she wanted, whenever she wanted. The only thing she didn’t seem to do was whoever she wanted. In fact, aside from The Kiss, she seemed positively Vestal.

As so, Elsa decided to take matters into her own hands. Kirk Davies, the Ravenclaw Head Boy, was the only student with the power to disable the wards in the Ravenclaw common room. As it so happened, he had been carrying a torch for Elsa for quite some time. And so she took advantage of that situation by allowing him to take advantage of her.

The Ravenclaw-Slytherin Quidditch match was that evening, and several professors had deliberately scheduled exams the following day in protest of this ridiculous new sport that was taking up more and more of the students’ attention. As such, the common rooms were completely empty. The students were either at the Quidditch pitch or in the library.

So under the cover of a few well-placed Webs of Darkness, he was able to sneak Elsa in through the portrait hole, and into the Ravenclaw common room. They both were practically shaking with excitement, for two very different reasons. She was so close. He was so close. All that needed to happen was for Kirk to slip upstairs to disable the wards on the sleeping chambers.

As he did so, Elsa very quickly, very deftly, pulled out her Quik-Sketch: an “academic assistance tool” that was invented a few decades back that was often used for less-than-honest purposes. It operated much like a modern-day magical camera, but it was uniquely suited to reproducing text. One could snap a quick sketch (hence the name) of a book, without even opening it, and later could peruse through the entire volume. This made it exceedingly easy for an unwitting Professor to have his or her entire curriculum surreptitiously copied and distributed.

And so, they were, for the most part, forbidden from classrooms, but as they had their uses, they weren’t entirely outlawed. After all, there were many one-of-a-kind tomes in the library, and it was much safer to have a student read through a Quik-Sketch than it was to entrust them with the physical tome itself.

Making sure to touch nothing, she quickly took a snapshot of the desk, making sure to contain the entire scene within the Quik-Sketch’s viewport. Not a moment too soon, she stowed it away in her robes, and Kirk emerged from the chamber. He had removed his robe, and beneath them, he was wearing pants and a collared shirt. She gave him an appraising look. He was handsome enough. Maybe this wouldn’t be too bad.

Or maybe it would be. As she endured the sloppy, uncomfortable encounter (after all, Kirk Davies was still somewhat new at this), she thought to herself that whatever was on those notes would have to be worth it. She could think of very little besides what new lore this might lead her to. Which was all for the best, because Kirk’s wild thrusting and ridiculous, syncopated gasps had obliterated any semblance of physical pleasure she may have been deriving from the experience.

Figuring she’d hurry things along a little bit, she dug her fingernails into his back, and whispered in the most sultry voice she could muster, “Oh, Kirk!”

As she did so, she forcefully rolled him onto his back and began to aggressively rock her hips back and forth in feigned ecstasy. She could feel him begin to tense and squirm, and within moments, she could feel that she had accomplished her task. She figured she’d give him a feeling of accomplishment as well, and so she let out a series of rhythmic moans, each a bit louder than the next.

She was proud of her performance.

He lay there, quivering on the bed, with a permanent grin plastered on his face. Even though she could clearly see the Bertoxxulous Ring and its distinctive purple halo hovering above the bed, there was no sense in not playing it safe. Normally, she would wait until she was in the restroom, behind closed doors, to cast the spell, but she might as well get it out of the way now. She pointed her wand about an inch and a half below her belly button and whispered.

“Animatus Mobilius Expelsor!”

She winced as the spray of liquid shot out of her. It was always a bit uncomfortable and inelegant, but it certainly beat the alternative, and besides, it was his mess to clean up now.

The Bertoxxulous Ring and the Parasite-Expelling Charm were a pair of spells that originally started off as defensive and healing charms, but had since been repurposed by the intrepid and inventive youth of Hogwarts to serve as a remarkably safe and effective means of birth control.

“The Ring” created an area-of-effect defensive aura which would prevent any life from reproducing underneath its halo. It was designed originally as a counter to the plant-based attacks that had been so in vogue a few centuries back. The “PEC” on the other hand, was designed to flush any unwanted entity from the body. It was remarkably effective against parasites such as Ceti Eels, Nargles, and, as it so happened, gametes.

One would have to be exceptionally careless, exceptionally unlucky, or exceptionally stupid to wind up with an unwanted child as a wizard. And even if you did, there were certain herbs and potions that could take care of that. Although they were not discussed as openly as The Ring and the PEC, there was little taboo concerning the subject.

There was no Wizarding equivalent to the debate of when, exactly, life begins. A simple Hominem Revelio would tell you, unequivocally, whether or not another life existed inside of you. They did not need to rigorously define the term “life” because Magic did it for them.

All of this was to say that the attitude towards sexual relationships at Hogwarts was quite cavalier, and such encounters rarely carried with them a great deal of emotional attachment. There were maybe twenty students per house per year, which did not make for many permutations of relationships, even taking into account that Hogwarts (and much of the Wizarding world) was far from heteronormative. Becoming attached and holding grudges was a dangerous proposition.

Nell had considered all of this, prior to doing Kirk Davies this favor. She had given some thought as to whether or not this would hurt Kirk because it would almost certainly be a one-time encounter, rather than the start of any sort of relationship. But that was probably for the best since Elsa probably wasn’t the best fit for Kirk.

The encounter itself was easy enough to facilitate. All Nell had to do was find something that Elsa wanted that only Kirk could provide. Kirk was Head Boy, and Elsa was ambitious, so it probably would have happened organically at one point or another. It was best that it happened under controlled conditions, and in such a way that Kirk would feel indebted to her.

As for Elsa’s part, she had retreated into the Slytherin chambers and had bribed a prefect to let her use an office. The set the Quik-Sketch up on a small stool, and it projected the interactive image of the notes across the desk. She shuffled through them, nudging pages out of the frame with her wand, rapidly scanning for keywords as she went on until she finally found something interesting:

“On Formally Indecipherable Incantations in Principia Discordia and Related Texts…”

This had to be it. Principia Discordia was famously obtuse, but it was considered one of the seminal works of first-order magical theory. Elsa didn’t quite grasp the finer points of it: after all, it did spend over 300 pages rigorously defining from first principles the fact that “Ma – Ha – Su” is not equal to “Su – Ma – Ha”.

They had studied the volume in one of her N.E.W.T.-level Magic Theory classes and she had learned just enough to pass the exam. The biggest takeaway was that all Magic could be fundamentally reduced to basic Axiomatic Forms. And because of this, Magic cannot self-reference. This lack of self-reference implied a lack of recursion, which in turn implied a vast number of laws and magical limitations. Gamp’s Law of Elemental Transfiguration, the Form/Substance Dichotomy, the Inverse Time-Complexity Relationship, and so on and so forth.

The tome did not seek to define precisely what these Axiomatic Forms were because for the most part, they are ineffable. But further to the point, there are an infinite number of potential Forms; they do not have to be ‘true’, necessarily, they simply have to be syntactically correct. By assuming the Axiomatic property of a Form, one can derive all manner of theoretical spells that would be possible, so long as the original assumption is correct.

For example, if one assumes that the two forms, “Ma” and “Ntok”, can be combined, then it logically follows that a modifier can be used, based on previously defined theorems. One can then take that a step further and deduce that the modifier must be numerical in nature, and fit within a certain set of vocal inflections. A scholar of languages would note that Japanese was likely the only known language that could imply the proper meaning while still falling within the necessary range of sounds.

As such, an entirely new spell and its effects could be unequivocally proven by the simple (but lengthy) process of magical deduction, if only one assumes a certain Axiomatic Form. Of course, therein lies the rub: how does one know which Forms are Axiomatic and which are not? Most advanced magical research involves taking existing spells and attempting to determine which principles must be true in order for those spells to be possible. And most spell creation involved combining existing Axiomatic Forms in new and novel ways in order to achieve the desired result.

Truly visionary, or perhaps truly dangerous wizards would venture into unknown territory, devising a fantastically powerful effect, working backward, and then simply hoping or praying that the underlying Forms required were, in fact, valid. Many Wizards lost their lives in horrifically violent fashion by venturing down this road. But others have succeeded, creating rituals of absurd power.

Which in and of itself, gave rise to one of the greatest debates of the past 500 years, the problem of Convenient Axioms. Why was it that certain Forms just “happened” to be Axiomatic? Almost all Axioms that had been discovered had some form of practical application. Why was this the case? It was, as the problem suggested, too convenient.

One faction argued that more than likely, there were countless more Axioms yet to be discovered that did not have any practical application, which was precisely why they had been undiscovered. After all, how could they be experimentally verified? The opposing faction cited several examples of possible means of validation, and further pressed the issue, citing the limitations of reductionism. Eventually, you hit the end of the line, and have to answer the question: “Where do these Axioms come from?”

Principia Discordia did not bother with such esoteric (or, as Whitehead’s opponents would say, “practical”) questions. Asking “why” is as asinine as asking why 1=1. The Law of Identity needs no proof; the simple act of considering the proof presupposes its validity. The world is what the world is, and only something fundamentally extra-worldly would have the power to create and define such Axioms. But even that supernatural force would be subject to its own set of Axioms and laws. And so on for any super-supernatural forces, and so on and so forth.

Whitehead’s crowning achievement, in his mind, was constructing a language of Magic so simple and rigorous that it could be extended forever upward, and given sufficient time, could enumerate all possible iterations of all possible Axioms.

Elsa, of course, did not care about any of that. She learned what she needed to learn in order to pass the class, and what she learned told her that the overly complex notes that Nell had laid out on the page, including something about “when preceded by its incantation”, well, she knew enough to know that it just wasn’t possible.

She was about to be sorely disappointed until she noticed something that did catch her attention: a sketch of a pair of diaries, along with accompanying notes, that had been crossed out angrily. By shining her wand’s light at just the right angle, she could see the indentations in the paper, and could deduce what the sketch had been meant to illustrate: it was a linkage between the two diaries. Anything written in the first book would show up in its twin.

That was… interesting.

She ignored the obvious question of why that would be necessary, and further scanned the notes. It was clear that Nell had created a prototype, hidden it within the castle, and then discovered that her attempt was unsuccessful. Because it was worthless, she had abandoned it, but Elsa could tell that it wasn’t necessarily a failure. She grasped enough of the theory to think that maybe, just maybe, it was fixable.

The book was hidden, of course, in the library, because a single book tucked away in a place that was obviously meant for hiding things would draw attention. But a single book tucked away on a packed, nondescript shelf in the Restricted Area, which was already filled with books that had traces of magic, well, no one would ever notice that.

No one except Elsa.

She discovered that the failed prototype had been enchanted with a modified Protean charm, and although the binding magic had been removed, the reference still existed in the traces of magic that were woven through the book. A little-known loophole in the Protean charm would allow one to recover that linkage, and then… Well, then what?

Elsa knew what a potential treasure trove this book represented. A continually up-to-date copy of the diary of Perenelle du Marais. The secrets, the hidden lore, the potential for blackmail, to the right person, this was priceless. Although, someone clever enough to see its value would also be prudent enough to question why such a backup diary was even necessary. The fact that it even existed suggested something dark, in and of itself.

Unfortunately, Elsa did not have enough experience with the practical ins and outs of Magic to figure out how to take the next step, which was fixing the book once the linkage to its twin had been recovered. And so, she went to visit the one person who almost certainly did have that experience, a person who probably would not have minded seeing Nell knocked down a peg or two.

It was during her office hours that Elsa had approached Baba Yaga and presented to her the notes and the diary and asked for suggestions. Without so much at glancing at the notes, and without even looking up at Elsa, Baba Yaga snapped, “These notes and this book do not belong to you, do they? Answer me truthfully, and you may stay. If you lie to me, I will know, and you will no longer be welcome in my office or classroom.”

“They’re Perenelle’s,” Elsa answered immediately. Baba Yaga looked up at her, interested.

“And what is your motivation?”

“I… I don’t like how she always seems to just… Win. I want to have something on her. I want to teach her what it feels like to lose.”

Baba Yaga said nothing.

After a few moments, Elsa self-consciously flushed and covered her mouth with her hands. “Oh God, what am I saying? I… I can’t believe I just said that out loud. Please, just… Just forget I came here. Never mind.”

As she was standing to leave, Baba Yaga held her hand up. “Sit down.” Elsa complied. “I, too, think that Ms. du Marais should learn to lose. Now, sit down, and let me see those.”

She roughly grabbed the Quik-Sketch, scanned through the drawing of the diary, looked at the real diary in Elsa’s hands, and let out a brief chuckle. With a slight flick of her finger, the diary began to warm, and emit a light golden glow, which died down after a few moments.

Elsa was taken aback. “It was that easy?”

“Everything is easy, child.”

Eager with anticipation, Elsa opened the cover. Words were appearing on the pages faster than she could read them. She flipped to a random page and began reading. She scanned quickly through to try to identify anything interesting, but wasn’t finding anything–

Well. Wasn’t this just something?

An entire passage devoted to describing, in lascivious detail, the physical beauty of Nell’s new Battle Magic professor. And describing in even more detail what Nell would let that new Battle Magic professor do to her. Elsa had to bite her lip to keep her from grinning. At the very least, this would be an embarrassment. She slowly lifted her eyes from the book to try to catch a glimpse of what Baba Yaga was doing.

What Elsa saw was startling. Baba Yaga seemed to have lost any interest whatsoever in the diary. Her eyes were darting furiously back and forth across the page of notes, faster than Elsa would have thought anyone was capable of even reading. The professor’s back was stiff, her muscles tensed, her posture coiled, ready to strike. Elsa recognized that page that Baba Yaga was fixated on. On Formally Indecipherable Incantations…

In an instant, Baba Yaga looked up, and locked eyes with Elsa. “Do not think about the elephant.”

“What?”

In that brief instant when their eyes had met and Elsa was busy thinking about an elephant, Baba Yaga was able to determine that Elsa did not comprehend the true meaning or implication behind these notes. She casually held her hand out. “Hand me that diary.”

Elsa reluctantly complied, and Baba Yaga quickly flipped to the last page of the book. On, drawn with angry red slashes of a quill, was an illustration of a line inscribed within a circle inscribed within a triangle. It took up the entire page. She began to laugh, and Elsa looked deeply uncomfortable.

Baba Yaga tossed the diary back to Elsa. “Take this as a token for your reward, and leave. Now.” Elsa was more than happy to comply and quickly turned on her heels to leave. The moment her back was turned, Baba Yaga parted her lips slightly, breathed out a wisp of Magic, and Elsa found that any memories of On Formally Indecipherable Incantations had been cleanly wiped from her mind
\simpleline
The diary was, of course, a fraud, the payload at the end of a deliberately laid trail of breadcrumbs, designed so that someone would find it. It served as both a means to an end and a fallback option in and of itself. She had written the passages in such a way that if any adult had suspected that Nell had been the unwitting victim of emotional and possibly sexual abuse from an authority figure, this diary would serve as complete confirmation.

There was nothing explicit, of course, because then people might suspect that the diary was written with the intent of being read. No, everything was hints, suggestions, implications, little turns of phrases or idiosyncrasies that would paint the picture to a clever reader. People always latched on much more strongly to conclusions that they came to themselves, and those that made them feel clever than those that were simply presented to them.

She had also worked with Ollie and Festivus to ensure that her mind contained clear traces of a traumatic experience being removed via Obliviation, which was simple enough. And she left other subtle hints here and there, to further reinforce the illusion. If she ever did need to enact her fallback plan, she would simply have a very public, very vocal panic attack in class. She would insist that it was just stress brought on by the exams, that there was nothing wrong.

In her sleep, though, she would be fitful. She would whisper… no… stop… don’t… And of course, when she awoke, she would vehemently deny that she whispered anything at all, and would become defensive and withdrawn when questioned about it.

This would raise several red flags amongst experienced, well-meaning adult wizards, who would then start to look for the signs of abuse. They’d start with a light probing of her mind, whereupon which they’d find the jagged telltales of Obliviation, which would not be evidence unto itself. They would also find the very recent, very vivid and emotional memory of her tearing pages from her diary and casting them into the fire in the Ravenclaw common room.

An investigation would reveal that the diary was linked and that a backup existed somewhere within the castle. Terrified of being uncovered, whoever had stolen it would return it anonymously to the authorities, who upon reading it would commence the witch hunt.

There were several flaws with this plan, which is why it was purely meant as a backup. In fact, the entire notion of her “bet” with Baba Yaga was just one large fallback plan. The fact that Baba Yaga was receptive to such a wager in the first place was information in and of itself and furthered Nell’s hypothesis. Her real plan was to simply raise the stakes, to invent a more interesting game than the one they were currently playing.

But even that, that too was just a fallback plan. Her true mission was to see for herself just how far the rabbit hole went. If there was someone more powerful than Baba Yaga, she would find that person. If there was someone more powerful than that, she would find them. She would tear down the gates of Heaven and confront God if that’s what it took to fix this broken world.

Fortunately, the world was broken in just the right places so as to make that path to God’s doorstep much easier to walk than one might expect. Like so many others, Nell had heard whispers of the Deathly Hallows. And as “luck” would have it, she strongly (and correctly) suspected that two of those Hallows were right under her nose here at Hogwarts. You can only chalk things up to coincidence so many times before you begin to suspect a hidden hand, and what better way to reveal that hand than to play right into it?

And so she did, which was the real reason behind her private audience with Ollie. His cousin happened to be Isabella Gaunt, eldest descendant of Celia Gaunt neé Peverell, and that angular, jet black stone inset into the ring on her finger, well, it fit all the patterns. Nell often wondered why she seemed to be the only one who noticed these sorts of patterns or asked these kinds of questions. But she had long since moved past being frustrated, however, realizing that simply taking action was most often the winning move.

Although Ollie offered, she did not want to steal the ring. It needed to be given, if even for only a short period of time. Isabelle knew that the ring possessed eldritch powers, but was unaware of the extent. She did have a deep interest in the more esoteric aspects of magic and was always looking for opportunities to collect more lore. The ring was doing her no good simply sitting on her finger and, being family, she trusted Ollie implicitly. Furthermore, she knew of Perenelle’s reputation and knew that she was no thief. This was a win-win.

Although the exchange of favor for favor was implied, it went unstated between Ollie and Nell, because no assurances were necessary. He had given her the ring at the start of their meeting, with no mention of any condition or request. They had a brief discussion about its properties, but Perenelle thought it was best to quickly move on to what it was that Ollie wanted.

She talked to him about Helena, answered his questions, and spoke of her friend’s deepest desires. She said nothing that was told to her in confidence, only things that the astute observer could deduce on their own. She spoke of the path he would need to walk in order to win her heart, what he should do, what he should not do, and so on and so forth.

But the true favor that she bestowed upon him was laying the trail of breadcrumbs that led him to the conclusion that this was not the path he should be walking. Unrequited, idealized, and idolized love is most often best left to the hallowed halls of one’s own imagination, and after a series of innocent questions regarding his plans for the future and his dreams of his life together with Helena, he slowly began to recognize this.

By the end of their conversation, he was openly weeping. He felt like a right idiot. He hadn’t been subtle, not in the slightest, and he could only imagine how awkward and uncomfortable he had made Helena feel. But he was ready to move on. He pulled Nell into a tight hug.

“If you breathe a word of this to Festivus… I’ll cut your toes off and feed them to Thestrals. He’d never let me live this down.”

Nell laughed and patted him gently on the back. “Well, I guess you better be super, extra nice to me, then shouldn’t you?”

“How could I not?” He separated from her and began to pack up his things. “I love you, you know. Not like, love, love, like that kind. But, like a friend. You’ve always been nice to me, even when other people haven’t, even when there’s not been any reason to.”

“I love you like that, too. And I’m nice to you because you’re the type of person who’s worth being nice to. Don’t ever change that.”

Ollie smiled. “I won’t.”
\simpleline
It was time for Nell to call in another favor. Typically she let them brew for a bit longer, but time was of the essence, in more ways than one. It was about a week since his conquest of Elsa Greengrass, and Kirk DaviEs was still walking on rainbows. By a happy coincidence, he happened to be close friends with Gregory Potter, who happened to be the oldest member of the branch of the Potter clan that descended from Iolanthe Potter neé Peverell.

Nell had once joked that there was actually no such thing as Magic, that it was all just one long string of entirely improbable coincidences, that all magical phenomenon would have occurred anyway, it’s just that they coincidentally occurred immediately after someone spoke a particular phrase or waved a stick about in a certain way, and that the best wizards were the ones who were best able to take advantage of coincidences when they came up.

Granted, she had come up with this theory during a post-exam celebration, after inhaling quite a large amount of particularly strong Longbottom Leaf. She had challenged several of her companions to disprove her theory, and being that they were in a similarly altered state, they found themselves unable to argue.

When she recovered the next morning with a very dry mouth and a voracious appetite, she noticed that she had jotted down almost an entire scroll worth of notes. Apparently, she had tried to formally prove her proposition, and in the process had made some particularly appalling leaps in logic and faith. But it was still amusing to consider from time to time, and she thought of that now as she took advantage of yet another unlikely coincidence.

Gregory Potter was almost as infamous as Festivus Weasley for his
troublemaking and general prankery, and it was not altogether shocking that the two of them did not get on well, at all. This was due in no small part to the Festivus’ longstanding suspicion that Potter had in his possession a secret weapon that gave him an unfair advantage in their battle of one-upmanship. After all, some of the stuff that Gregory had pulled off simply couldn’t be done without an Invisibility Cloak.

When Kirk Davies had approached him, Gregory was hesitant, at first. The Cloak had to be worth a fortune, plus it was a family heirloom. But once he learned that the request was on behalf of Perenelle du Marais, he practically fell over himself to get to his trunk and hand it over. Not only would she owe him one, but just think how much this would get under Festivus’ skin!

Gregory was always annoyed at how that fire-headed git followed around Perenelle like a lost puppy. Didn’t he have any sense of pride? Not that he wouldn’t have a go at her, he’d certainly done worse, much worse. But he would never chain himself to a girl the way Festivus did, and so he was maliciously delighted at the idea of her being in debt to him.

As he shoved the Cloak into Kirk’s arms, he reminded him. “I’m not giving this to you, it’s just a loan.”

“Thanks again, buddy.”

“Don’t mention it. Actually, do mention in. Make sure Weaselby knows just how much I was able to help out his girlfriend.” Potter grinned.

“Uhh… You know, I’m pretty sure she doesn’t–”

Gregory waved him off. “Yeah, yeah, I know. That’s what makes it all the more depressing.”

“Well, I owe her one. I don’t know how she did it, but she set me up good. Reaaaaaallll good. Have I told you about what happened with Elsa?”

Potter rolled his eyes. “Elsa? You mean, Elsa Greengrass? The same Elsa Greengrass that you spent a night with and have been telling everyone in the school five times a day ever since? That Elsa?”

“Yep. That one. So you’ve heard the story then?”

“Only about a hundred times.”

“Well, one more time can’t hurt.” Kirk continued as Potter groaned. “So there we were, in the prefect’s bedroom, and she starts doing this crazy, corkscrewy sorta thing…”
\simpleline
At no point did Nell ever think that the fact that she was only in her sixth year should be any detriment to her whatsoever. The scale of power in the Wizarding world was exponential, not linear. The types of people she was seeking out, they were orders of magnitude more powerful than her, or any of her classmates, or for that matter, any of her teachers. To think that she should wait a few more years until she was a bit more studied and a bit more powerful was like a spider thinking that if only it could grow a little bit bigger, a little bit stronger, then it could fight a dragon in hand-to-hand combat.

She had stolen that analogy from one of the essays and lessons that Baba Yaga had assigned to their class. “You have been permanently transfigured into a spider. You still retain your human intelligence and human lifespan. In the mountain nearby lives a Hungarian Horntail. Describe how you would defeat it.”

The next day in class when they had turned in their essays, Baba Yaga had revealed to them a specialized device she had created for this exercise. To grade the essays, she would read them aloud into the device, which would then simulate the scenario and proposed solution ten thousand times over, assigning a point for each favorable outcome.

It very neatly illustrated several key concepts in Battle Magic. Firstly, it showed the limitations of cleverness when faced with brute strength: even the most effective solution amongst the class was successful only about a third of the time.

“In every battle, there is a dragon and there is a spider, and your tactics and strategy must differ depending on your role. Many Wizards have met their doom because they were spiders convinced they were dragons. And similarly, many Wizards have consigned themselves to lifetimes of frustration because they were dragons subjecting themselves to the limitations of spiders. Know who you are, know your role, and fight accordingly.”

But there was another lesson that was just as important. The worst essay in the class turned in by a very much hungover and still slightly drunk Randall Flaggstone, simply stated, “Sneak up on the dragon and bite him in the eye.”

This worked, three times out of ten thousand.

“No matter how powerful you are, there is still the possibility that your opponent’s plans will succeed due to sheer, dumb luck. This is why you rarely hear of many Dark wizards or witches who last beyond a generation or two. There are more spiders in this world than there are dragons, and ten thousand spiders with ten thousand idiotic ideas can and will one day bring you down. The lesson here is simple: do not give spiders a reason to attack.”

At this lesson, Nell had interrupted. “But there are some in the world who would hate you for being good, resent your purity and your goodness. There really are people like that in the world, you know.”

Baba Yaga smiled, cruelly. “Yes, believe me, I know. Perhaps this is also why you do not hear of very many Light wizards who live beyond a generation or two. No, the trick to self-preservation is to be lukewarm; neither hot nor cold, neither good nor evil.”

“There are some in the world who would hate you for that, even.” Nell responded, coolly.

“I’m willing to take my chances.”
\simpleline
Cloak and Stone in hand, Nell had gone about the task of trying to slay a dragon. She cast her mind across the Cloak, examining its properties, trying to consider precisely what differentiated it from a typical Cloak of Invisibility, despite mere longevity. It repelled the eyes, yes, but that functionality was almost… ancillary. It really was like two Cloaks in one. The outer layer which kept the wearer unseen, and the inner layer which kept the wearer hidden.

It seemed unaffected by ambient Magic, which passed through it as easily as did light. But targeted, direct magic behaved differently, if not unreliable. It took them several iterations of tests to finally determine with sufficient confidence the nature of the Cloak.

In essence, it kept you Hidden from Magic, so long as you remained Unseen. A curse would travel right through you if its caster did not truly know you were there. As with all things magical, there seemed to be a very fuzzy line between knowledge and belief, with the power of the effect seeming to be directly proportionate to the strength of the conviction.

She had worked together with Helena to try to figure things out. Nell had hidden in a corner, but then snapped her fingers to reveal her presence. Helena’s stunning bolt hit her with full force. On the other hand, when she told Helena that she would hide in one of four places when she was finally struck by the bolt, it felt much less powerful. Nell was inclined to say it felt a quarter as powerful.

So they did more experiments. Nell told Helena she would be hiding in the corner and told her to enter the room and fire there. In truth, Nell stood immediately in front of the doorway, directly in the line of fire of the curse. The bolt went right through her.

They repeated the exact same trick again, but this time the bolt stunned her, although it was relatively weak. When she questioned Helena about it, she said, “Well, I kind of thought you might try the same thing again, but I wasn’t sure.” The effect seemed to live somewhere at the intersection of belief and reality. It wasn’t enough for one to be correct, but it also wasn’t enough for one to simply believe.

That was when she realized it, the need for secrecy, why it was so important. The more people who knew that you possessed the Cloak, the more likely it was that someone could make an educated guess and be correct. If you went traipsing about the school like that idiot Potter boy had done, it really wouldn’t do you much good because even Festivus Weasley could figure out your secret. But if no one knew you were there, and no one knew you were coming, you could stay hidden, perhaps, forever.
\simpleline
\DatePlace{Alderney\\
June 12, 1334, C.E.}
Cadmus was getting ready to retire for the night. He had eaten more than his fill of wild game and drank more than his fair share of wine. It was time to sleep. Or at least, it would have been, had he not sensed something awry. He briefly considered drawing the Wand but decided it wasn’t worth it. The overwhelming majority of threats could be dealt with without resorting to that. And besides, it simply wasn’t worth the risk. Although the definition of “defeat” was remarkably fickle, if he did not use the Wand, then he would not have to risk losing it.

Instead, he shuffled awkwardly over to his study, which contained an entire wall of Dark Detectors of various shapes and sizes and mechanisms.

They were all motionless.

Odd. His intuition was rarely wrong, but then again, the Dark Detectors were rarely wrong either. He designed them himself, after all. He opened up a glass case and removed the Eye of Vance, and peered through it.

Nothing.

Any fogginess brought about by the drink was counteracted by the adrenaline that was coursing through Cadmus. It seemed like a false alarm, but he could never be too cautious. He put off his plans of going to bed, and instead, sat in his chair in his study, and began to read to pass the time, making sure to keep one eye on the Dark Detectors.

Another hour or so passed, and he could feel himself begin to doze off. The feeling of apprehension had passed, and so perhaps it was now safe. He stood up, stretching, and his considerable girth began to weigh on his joints. He closed his book, walked towards the exit of his study, and that’s when he heard the noise.

The front door slammed open, and in through walked… no, not walked… floated a familiar figure, translucent, wavering, grave. Cadmus felt sick to his stomach.

The spirit of Ignotus Peverell neé Hand beckoned to him, and spoke. “SHE IS HERE. THE ONE WHO WILL TEAR APART THE VERY STARS IN HEAVEN. SHE IS HERE. SHE IS THE END OF THE WORLD.”

In one swift movement, Cadmus withdrew the Elder Wand, the Deathstick, the Wand of Destiny, and then all was darkness.

